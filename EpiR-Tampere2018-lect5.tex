% typeset using LaTeX2e
%\documentclass[handout, 12pt]{beamer}
\documentclass[handout,12pt]{beamer}
%\mode<presentation>
%{
% \usetheme{Singapore}
% \setbeamercovered{transparent}
%}

% making handouts
%\usepackage[overlay]{textpos}
%\usepackage{pgfpages}
%\pgfpagesuselayout{4 on 1}[letterpaper, border shrink = 5mm, landscape
%]
%\mode<presentation>{
% 
%}
% end of making handouts

\usepackage[latin1]{inputenc}
\usepackage[english]{babel}
\usepackage{epsfig}
%\usepackage{rotating}
\usepackage{graphicx}
%\usepackage{mslapa}
\usepackage{amsmath}
%\usepackage[all]{xy}
% \usepackage{amssymb,graphicx}
%\input psfig.sty
\usepackage{booktabs}
\usepackage{graphpap}
\usepackage{verbatim}
%\usepackage{amsbsy}
%\graphicspath{{C:/Users/janne.pitkaniemi/Kingstontikku/figures/}}
\graphicspath{{C:/Users/janne.pitkaniemi/figures/}}

\setbeamertemplate{footline}[page number]
\DeclareGraphicsExtensions{.pdf, .jpg, .png,.jpeg}
% \DeclareGraphicsExtensions{.eps}

\newcommand{\Rlogo}[1]{\includegraphics[#1]{Rlogo}}

\input{usefulesa}
\usepackage[labelformat=empty]{caption}
\captionsetup[figure]{labelformat=empty}

%\newcommand{\R}{\bf \textsf{R}}
\parskip\medskipamount

\title{Epidemiologic Data Analysis using R \\
Part 5: Time-splitting and SIR}  % : \\ Analysis of Follow-up Studies}
\author{Janne Pitk\"aniemi \\ (Esa L{\"a}{\"a}r{\"a})}

\institute{Finnish Cancer Registry, Finland,   
 \texttt{<janne.pitkaniemi@cancer.fi>} \\
 (University of Oulu, Finland,   
 \texttt{<esa.laara@oulu.fi>}) }

\date{University of Tampere \\Faculty of Social Sciences \\ % University of Tampere/Postgraduate training,
Feb 26- Apr 9  2018}

%\begin{center} \Rlogo{height=2em} \end{center}

\AtBeginSubsection[]
{
  \begin{frame}<beamer>
    \frametitle{Outline}
    \tableofcontents[currentsection,currentsubsection]
  \end{frame}
}

%\beamerdefaultoverlayspecification{<+->}



\begin{document}

\definecolor{darkgreen}{rgb}{0,.5,0}

\begin{frame}
    \titlepage
\end{frame}



%-------------------------------------------

\begin{frame}
\frametitle{Contents}
 
\bi
\item[1.] Special exposure cohorts 
\medskip
\item[2.] Observed and expected cases, SIR \& SMR
\medskip
\item[3.] Lexis diagram and life lines
\medskip
\item[4.] Splitting follow-up time simultaneously by age and period 
\medskip
\item[5.] Merging reference rates with the cohort data
and performing SIR/SMR computations 
\ei

Main R functions to be covered
\bi
\item {\tt Lexis.diagram()} and other {\tt Lexis} tools in {\tt Epi}  
\item {\tt merge()}
\ei
\end{frame}
%---------------------------------------

%---------------------------------------------------------
\begin{frame}
 \frametitle{Special cohorts of exposed subjects}
 
\bi
\item
Occupational cohorts, exposed to potentially \\
hazardous agents
\medskip
\item
Cohorts of patients on chronic medication, which may have harmful long-term side-effects
\ei
No internal comparison group of unexposed subjects.

\medskip
{\it Question}:
Do incidence or mortality rates in the \\ {\it exposed} target cohort differ from those of a
roughly comparable \textit{reference} population?

Reference rates obtained from: 
\bi
\item
 population statistics (mortality rates)
 \item
  disease \& hospital discharge registers (incidence)
  \ei
\end{frame} 
%----------------------------------------------------------------------
\begin{frame} \frametitle{Accounting for age distribution}

\bi
\item Compare rates in a study cohort with a standard set of
  age-specific rates from the reference population.
  \medskip
\item Reference rates normally based on large numbers of cases, so
  they are assumed to be ``known'' without error.
  \medskip
\item Calculate {\bf expected} number of cases, $E$, if the standard
  age-specific rates had applied in our study cohort. 
 \medskip
\item  
  Compare this with the
  {\bf observed} number of cases, $D$, by the {\bf standardized incidence ratio} SIR \\
  (or standardized mortality ratio SMR)
$$ \text{SIR} = D/E, \qquad \text{SE}[\log(\text{SIR})] = 1/\sqrt{D} $$

\ei

\end{frame}
%----------------------------------------------------------------

%----------------------------------------------------------------------
\begin{frame} \frametitle{Example: HT and breast ca.}
\bi
\item
A cohort of 974 women treated with hormone (replacement) therapy
 were followed up. 
 \item
$D$ = 15 incident cases of breast cancer were
observed. 
\item 
Person-years ($Y_a$) and reference rates 
($\lambda_a^*$, per 100000 y)
by age group ($a$) were:
\ei
\small
\begin{center}
\begin{tabular}{lrrr}
Age & $Y_a$   & $\lambda_a^*$ & $E_a$ \\ 
\hline
40--44 \rule{0em}{1em} &  975 & 113 & 1.10 \\
45--49 & 1079 & 162 & 1.75 \\
50--54 & 2161 & 151 & 3.26 \\ 
55--59 & 2793 & 183 & 5.11 \\
60--64 & 3096 & 179 & 5.54 \\
\hline 
$\sum$ \rule{0em}{1em} & & & 16.77
\end{tabular}
\end{center}
\end{frame}

\begin{frame}
\frametitle{Example: HT use and breast ca. (cont'd)}

\begin{itemize} 
\item ``Expected'' number of cases at ages 40--44:
\[ 975 \times \frac{113}{100\,000} = 1.10 \]
\medskip
\item Total ``expected'' cases is $E = 16.77$
 \medskip 
\item The SIR is $15/16.77 = 0.89$.
 \medskip 
\item Error factor: $\exp(1.96\times \sqrt{1/15}) = 1.66$
 \medskip 
\item 95\% confidence interval is:
 \[ 0.89 \mpydiv 1.66 = (0.54, 1.48) \]
\end{itemize}


\end{frame} 


\begin{frame}
\frametitle{A statistical model for SIR}
\ \\
\bi
\item
The theoretical rates $\lambda_{ap}$ by age ($a$)
and calendar period ($p$) in the cohort are assumed to be  
proportional to the rates $\lambda^*_{ap}$ in
the reference population:
$$
   \lambda_{ap} = \rho\times \lambda^*_{ap}  
$$
\item[ ]
$\rho$ = hazard ratio btw the cohort and the reference
pop'n. 
\medskip
\item
The population rates 
$\lambda^*_{ap}$ are assumed to be known.
\medskip
\item
Cohort data: numbers of cases $D_{ap}$ and p-years $Y_{ap}$ by $a$ge and $p$eriod are computed.
\medskip
\item
It can be shown that the likelihood of $\rho$ is of Poisson type, and the maximum likelihood estimator of $\rho$ is:
\[
  \widehat\rho = \frac{D}{\sum \lambda^*_{ap} Y_{ap} } =
  \frac{\mbox{Observed}}{\mbox{Expected}}
  = \text{SIR}
\]

\ei
 
\end{frame}

%
%%----------------------------------------------------------------------
%\begin{frame} \frametitle{Log-likelihood for rate ratio $\theta$}
%\bi
%\item[ ]
%\bes
%  \ell(\theta) & = & \sum \left[ D_{ap} \log( \lambda_{ap} ) - \lambda_{ap} Y_{ap} \right] \\
%               & = &
%   \sum \left[ D_{ap} \log( \rho \lambda^*_{ap} ) - \rho\lambda^*_{ap} Y_{ap} \right] \\
%     &  = &
%   \sum \left[ D_{ap} \log( \rho ) - \rho(\lambda^*_{ap} Y_{ap}) \right] \\
%    & = & D \log( \rho ) - \rho \sum \lambda^*_{ap} Y_{ap} 
%\ees
%omitting terms which do not depend on
%$\rho$.
%\medskip
%\item
%Similar to the Poisson type of log-likelihood for a 
%theoretical ``rate'' $\rho$, based on an
%empirical ``rate'' $D/\sum \lambda^*_{ap} Y_{ap}$.
%\medskip
%\item
%Thus the maximum likelihood estimator of $\rho$ is:
%\[
%  \widehat\rho = \frac{D}{\sum \lambda^*_{ap} Y_{ap} } =
%  \frac{\mbox{Observed}}{\mbox{Expected}}
%  = \text{SIR}
%\]
%% $\SMR$ is the ML estimator of the incidence (mortality) rate ratio
%% between the cohort and the reference population.
%
%\ei
%
%\end{frame}
%

\begin{frame}
\frametitle{Example: The Welsh Nickel Workers' Study}

\bi
\item
A cohort of 679 men working in nickel smelters in South Wales first employed 1903-25 (for details see {\bf B\&D}). 
\medskip
\item
Outcomes of interest: deaths from nasal (ICD code 160) 
and lung cancer (ICD 162 and 163) during follow-up 1934-76. 
\medskip
\item 
Outcome event indicator and basic time variables:
\bes
{\tt icd} & = & \mbox{code for cause of death, 0 if not yet dead}\\
{\tt date.bth} & = & \mbox{date of birth}\\
{\tt date.in} & = & \mbox{date of starting follow-up}\\
{\tt date.out} & = & \mbox{date when follow-up ended}
\ees 
\ei

\end{frame}

%-----------------------------------

\begin{frame}
\frametitle{Example (cont'd)}

\bi
\item
Interesting risk factors in the original data frame:
\bes
\mbox{\tt expos} & = & \mbox{exposure index based on years employed in}\\
  & { } & \mbox{high-risk areas in the smelter by 1925}\\
  & { } & \rightarrow \mbox{ categorized version EXP} \\
%  & { } & 1\ : \mbox{ 0 y}\\
%  & { } & 2\ : \mbox{ 0.5 to 4.0 y}\\
%  & { } & 3\ : \mbox{ 4.5 to 8.0 y}\\
%  & { } & 4\ : \mbox{ 8.5 to 12.0 y}\\
%  & { } & 5\ : \mbox{ $\geq$ 12.5 y}\\
\mbox{\tt date.1st} & = & \mbox{date when first employed }
     \rightarrow\mbox{ AFE}
\ees
\item
Risk factors to be formed from original variables:
\bes
\mbox{\tt age.1st} & = & \mbox{age when first employed }
     \rightarrow\mbox{ AFE} \\
\mbox{\tt year.1st} & = & \mbox{year of first employment }
   \rightarrow \mbox{ YFE} \\
\mbox{\tt time.1st} & = & 
\mbox{time since first exposure }\rightarrow\mbox{ TFE}
\ees
\ei
\end{frame}


%----------------------------------------------------------------------
\begin{frame} 
\frametitle{Lexis diagram \& 4 lifelines from the nickel cohort}
\ \\
Diagram invented by {\it Wilhelm Lexis} (1837-1914), 
German mathematician and demographer, professor in
Tartu 1874-76.

\begin{columns}[t]
\begin{column}{0.45\textwidth}
\begin{center}
\includegraphics[height=2in, width=2in]{Lexis1}
\end{center}
\end{column}
\begin{column}{0.45\textwidth}

Individual lifelines run dia-\\
gonally from 
a given\\
 (age, time) starting point\\
to an endpoint. 

\bigskip
Here the lines go from \\
start of exposure till \\
 the age and time of exit.
 
 \bigskip
 Mortality follow-up started in 1934.
\end{column}
\end{columns}

\end{frame} 



%-------------------------

\begin{frame}
 \frametitle{Nickel cohort: All lifelines in the Lexis diagram}
 
\begin{columns}[t]
\begin{column}{0.45\textwidth}
 \begin{center}
\includegraphics[height=2in]{Lexis2}
\end{center}
\end{column}
\begin{column}{0.45\textwidth}

\bigskip
\bigskip
Follow-up starts not until 1934 for all subjects.

\bi
\item
dot (red) \\ = lung ca. death, 
\item
circle (green)\\  = censoring
\ei  
\end{column}
\end{columns}

Function {\tt splitLexis()} splits individual follow-up times into
rectangles defined by agebands and calendar periods. 
\end{frame} 



%----------------------------------------------------------------------
\begin{frame} \frametitle{Splitting follow-up by age \& calendar time}

\begin{columns}[t]
\begin{column}{0.6\textwidth}
{\bf from} the registration of:
\bi
\item   Entry,
\item  Exit,
\item Failure status
\ei
 of the individuals in the cohort,
 and the definition of the scale by:
 \bi
 \item
 \textbf{O}rigin
 \item
 \textbf{S}cale
 \item
 \textbf{C}utpoints\\
\ei
\end{column}
%
\begin{column}{0.4\textwidth}
\bigskip
{\bf to} the table of:
\bi
\item
  $D$ = events,
  \medskip
  \item
  $Y$ = person time,
  \ei
 by age and period.
\end{column}
\end{columns}

\end{frame} 

%----------------------------------------------------------------------
\begin{frame} \frametitle{Expected numbers in practice}
\bi
\item From the records of age-period split \& expanded 
  cohort data:
\item[ ] $\quad y_{i,ap}$ = person-time slot in a record defined by
\item[ ] $\quad a$ = ageband of the record
\item[ ] $\quad p$ = period of the record
% \item sex = The sex of the record

\bigskip

\item From the file containing the reference rates:
\item[ ] $\quad \lambda_{ap}^*$ = age \& period specific rate
\item[ ] $\quad a$ = ageband of the population rate
\item[ ] $\quad p$ = period of the population rate
% \item sex --- The sex of the population rate
\ei
\end{frame}


\begin{frame}
\frametitle{Expected numbers in practice (cont'd)} 
\ \\

 Population rates are matched up 
to the expanded cohort data, and
  expected numbers individually are computed as:
$$ e_{i,ap} = \lambda_{ap}^* \times y_{i,ap}  $$
and these are eventually summed: $E = \sum e_{i,ap}$

\bigskip
Always two datasets are needed for SIR:
\bi
\item[1.]
 the \textit{cohort} data with follow-up information on its
  individual members. This must be split \& expanded to match with
\item[2.] the \textit{reference rate} data with age \& period specific rates in the chosen reference population.
\ei

\end{frame}


%---------------------------------------------------------

\begin{frame}[fragile]
 \frametitle{SMR-calculations in \R \ using Lexis tools:} 
 
 \ \\
{\large \color{blue} 
  1. Read in the cohort data (Welsh Nickel Workers) }
  
  \medskip
 and  convert the dates \texttt{dd/mm/yyyy} into ``decimal years'' 
{\small
\renewcommand{\baselinestretch}{0.9}
\begin{verbatim}
> nick <- read.table( "nickel.txt", 
+        header=T, as.is=T )
> for (j in 4:7 ) nick[ , j] <-
+   cal.yr( nick[ , j], format = "%d/%m/%Y" )
\end{verbatim}
}
List the records for the 4 men in a previous Lexis diagram
{\small
\begin{verbatim}
> round(nick[11:14, ],2)
   id icd expos date.bth date.1st date.in date.out
11 19 160  10.0  1881.73  1915.18 1934.25  1940.21
12 21  14   0.0  1877.80  1908.00 1934.25  1943.37
13 22 177   2.5  1879.50  1908.17 1934.25  1946.98
14 23 162   0.0  1900.50  1923.15 1934.25  1953.20
\end{verbatim}
}

\end{frame}

\begin{frame}[fragile]
 \frametitle{2. Reference rates in E \& W read in}
{\small
\renewcommand{\baselinestretch}{0.9}
\begin{verbatim}
> ewrates <- read.table("ewrates.txt",header=T)
> ewrates[c(1:8, 143:150), ]  
\end{verbatim}
\renewcommand{\baselinestretch}{1.0} 
}
8 first and last rows checked
{\scriptsize
\renewcommand{\baselinestretch}{0.9}
\begin{verbatim}
    year age lung nasal  other
1   1931  10    1     0   1269
2   1931  15    2     0   2201
3   1931  20    6     0   3116
4   1931  25   14     0   3024
5   1931  30   30     1   3188
6   1931  35   68     1   4165
7   1931  40  149     3   5651
143 1976  45  403     3   4311
144 1976  50 1003     9   7687
145 1976  55 1896     9  12544
146 1976  60 3342    15  20787
147 1976  65 4985    17  33729
148 1976  70 6718    20  55480
149 1976  75 8068    38  89199
150 1976  80 7744    33 137360

\end{verbatim}  
\renewcommand{\baselinestretch}{1.0} 
}

\end{frame}

\begin{frame}[fragile]
\frametitle{E \& W lung ca. death rates by age and period}
{\small
\renewcommand{\baselinestretch}{0.9}
\begin{verbatim} 
> tapply(ewrates$lung, list("age" = ewrates$age,
         "year" = ewrates$year),sum)
\end{verbatim}
\renewcommand{\baselinestretch}{1.0} 
}
{\scriptsize
\renewcommand{\baselinestretch}{0.9}
\begin{verbatim}
    year
age  1931 1936 1941 1946 1951 1956 1961 1966 1971 1976
  10    1    1    1    1    0    0    0    0    0    0
  15    2    2    2    3    2    2    2    2    1    1
  20    6    6    6    8    7    4    5    4    4    2
  25   14   14   16   18   13   12   11   10   10    7
  30   30   30   34   36   35   35   34   25   24   17
  35   68   68   81   94   98   93   90   76   58   56
  40  149  149  191  236  248  251  223  216  177  139
  45  274  274  384  544  579  590  563  531  503  403
  50  431  431  597  954 1224 1248 1221 1160 1070 1003
  55  586  586  883 1350 2003 2317 2284 2201 2077 1896
  60  646  646 1021 1717 2555 3315 3663 3695 3546 3342
  65  636  636  970 1763 2926 3926 4844 5273 5174 4985
  70  533  533  748 1400 2624 3878 4977 6210 6820 6718
  75  464  464  631 1085 2069 3332 4513 5914 7273 8068
  80  324  324  385  765 1416 2258 3417 4563 6089 7744
\end{verbatim}  
\renewcommand{\baselinestretch}{1.0} 
}

\end{frame}



\begin{frame}[fragile] 
\frametitle{3. Creating and expanding the \texttt{Lexis} object}

\ \\
The data frame converted to a \texttt{Lexis} object in \\
two time scales: \texttt{year} (calendar time) and \texttt{age}: 

{\small 
\renewcommand{\baselinestretch}{0.9}  
\begin{verbatim}  
nickL <- Lexis( entry = list( year = date.in ),
                 exit = list( year = date.out, 
            age = date.out - date.bth ),
    exit.status = as.numeric( nick$icd %in% c(162, 163) ),
           data = nick )
\end{verbatim}
}
The \texttt{Lexis} object jointly split by age and period. Agebands and period bands are named like in the \texttt{ewrates} file --
\texttt{"left"} means the lower cutpoint (1st year) of a band.
\small
\begin{verbatim}
> nickL.a <- splitLexis(nickL, "age", br=seq(10,85,5) )
> nickL.ap<- splitLexis(nickL.a,"year",br=seq(1931,1981,5))
> nickL.ap$year <- timeBand(nickL.ap, "year", "left")
> nickL.ap$age <- timeBand(nickL.ap, "age", "left")
\end{verbatim}  
\renewcommand{\baselinestretch}{1.0} 
%\normalsize

\end{frame}



\begin{frame} [fragile]
\frametitle{The expanded data frame viewed}
\small
\renewcommand{\baselinestretch}{0.9}  
\begin{verbatim}
> dim(nickL.ap)
[1] 6948   13    # 10-fold expansion!
> round( subset( nickL.ap, lex.id %in% 13:14)
+       [ , c(1:4,6,8,10,12,13)] ,2)
\end{verbatim}
\scriptsize
\begin{verbatim} 
    lex.id year age lex.dur lex.Xst icd date.bth date.in date.out
90      13 1931  50    0.25       0 177   1879.5 1934.25  1946.98
91      13 1931  55    1.50       0 177   1879.5 1934.25  1946.98
92      13 1936  55    3.50       0 177   1879.5 1934.25  1946.98
93      13 1936  60    1.50       0 177   1879.5 1934.25  1946.98
94      13 1941  60    3.50       0 177   1879.5 1934.25  1946.98
95      13 1941  65    1.50       0 177   1879.5 1934.25  1946.98
96      13 1946  65    0.98       0 177   1879.5 1934.25  1946.98
97      14 1931  30    1.25       0 162   1900.5 1934.25  1953.20
98      14 1931  35    0.50       0 162   1900.5 1934.25  1953.20
99      14 1936  35    4.50       0 162   1900.5 1934.25  1953.20
100     14 1936  40    0.50       0 162   1900.5 1934.25  1953.20
101     14 1941  40    4.50       0 162   1900.5 1934.25  1953.20
102     14 1941  45    0.50       0 162   1900.5 1934.25  1953.20
103     14 1946  45    4.50       0 162   1900.5 1934.25  1953.20
104     14 1946  50    0.50       0 162   1900.5 1934.25  1953.20
105     14 1951  50    2.20       1 162   1900.5 1934.25  1953.20
\end{verbatim}  
\renewcommand{\baselinestretch}{1.0} 
\normalsize

\end{frame}


\begin{frame} [fragile]
\frametitle{4. Merging the cohort data with E\&W rates} 
{\small
\renewcommand{\baselinestretch}{0.9}  
\begin{verbatim}   
> nickLew.ap <- merge(nickL.ap, ewrates, 
        by = c("age", "year"))  # key columns
> round(nickLew.ap[1:20, c(1:4,6:8,10,12,13,14) ],1) 
\end{verbatim}  
\renewcommand{\baselinestretch}{1.0} 
}
{\scriptsize
\renewcommand{\baselinestretch}{0.9}  
\begin{verbatim}   
    year age lex.id lex.dur lex.Xst  id icd date.bth date.in date.out lung
1  1931  20    197     0.3       0 273 154   1909.5  1934.2   1965.4    6
2  1931  20    236     1.3       0 325 434   1910.5  1934.2   1953.5    6
3  1931  20    400     0.5       0 574 491   1909.7  1934.2   1980.4    6
4  1931  20    384     0.3       0 546   0   1909.5  1934.2   1982.0    6
5  1931  20    156     0.9       0 213 162   1910.1  1934.2   1973.2    6
6  1931  25    236     0.5       0 325 434   1910.5  1934.2   1953.5   14
7  1931  25     38     0.3       0  56 502   1904.5  1934.2   1956.1   14
8  1931  25    581     1.5       0 842 420   1905.7  1934.2   1973.9   14
9  1931  25    267     0.1       0 369 420   1904.3  1934.2   1974.7   14
10 1931  25    478     1.8       0 690 420   1906.5  1934.2   1961.6   14
11 1931  25    251     1.8       0 344 420   1908.9  1934.2   1977.2   14
12 1931  25    156     0.9       0 213 162   1910.1  1934.2   1973.2   14
13 1931  25    400     1.3       0 574 491   1909.7  1934.2   1980.4   14
14 1931  25    390     1.8       0 556   0   1908.6  1934.2   1982.0   14
15 1931  25     85     1.0       0 111   0   1905.2  1934.2   1982.0   14
16 1931  25    315     1.1       0 443 420   1905.3  1934.2   1971.1   14
17 1931  25    168     0.1       0 227   0   1904.3  1934.2   1982.0   14
18 1931  25    169     1.8       0 228 502   1906.9  1934.2   1978.6   14
19 1931  25    121     1.5       0 157 332   1905.8  1934.2   1980.5   14
20 1931  25     17     1.6       0  28 420   1905.8  1934.2   1967.4   14
\end{verbatim}  
\renewcommand{\baselinestretch}{1.0} 
}

\end{frame}

\begin{frame}[fragile]
 \frametitle{5. Calculation of observed and expected}
\ \\
Cases \& person-time slots renamed, expectations $\lambda_{ap}^* y_{i,ap}$ of becoming a case computed,
and tables by $a$ and $p$ produced.
{\small
\renewcommand{\baselinestretch}{0.9}  
\begin{verbatim}
> nickLew.ap <- transform( nickLew.ap, 
+          d_iap = lex.Xst, y_iap = lex.dur, 
+          e_iap = lex.dur * lung/1.0E6 )

> Obs.lung <- with(nickLew.ap, tapply(d_iap,
+   list("age" = age, "year" = year), sum))

> Exp.lung <- with(nickLew.ap, tapply(e_iap,
+   list("age" = age, "year" = year), sum))

> Obs.lung ; round(Exp.lung,3)
\end{verbatim}  
\renewcommand{\baselinestretch}{1.0} 
}

\end{frame}

\begin{frame} [fragile]
\frametitle{Observed and expected numbers printed}
 
{\scriptsize
\renewcommand{\baselinestretch}{0.9}  
\begin{verbatim} 
    year      
age  1931 1936 1941 1946 1951 1956 1961 1966 1971 1976
  30    0    0    0   NA   NA   NA   NA   NA   NA   NA
  35    0    0    0    0   NA   NA   NA   NA   NA   NA
  40    0    1    1    0    0   NA   NA   NA   NA   NA
  45    3    2    4    1    0    0   NA   NA   NA   NA
  50    1    5    3    7    6    2    0   NA   NA   NA
  55    0    5    6    6    4    5    1    0   NA   NA
  60    1    4    5    3   11    6    1    1    1   NA
  65    0    0    1    5    4    6    3    1    0    0
  70    0    0    1    3    0    2    2    1    0    0
  75   NA    0    0    0    0    1    1    2    1    3
  80   NA   NA    0    0    0    0    1    0    0    3
    year
age   1931  1936  1941  1946  1951  1956  1961  1966  1971  1976
  30 0.004 0.005 0.001    NA    NA    NA    NA    NA    NA    NA
  35 0.012 0.032 0.015 0.004    NA    NA    NA    NA    NA    NA
  40 0.027 0.075 0.090 0.045 0.011    NA    NA    NA    NA    NA
  45 0.054 0.135 0.184 0.246 0.110 0.025    NA    NA    NA    NA
  50 0.082 0.231 0.281 0.438 0.511 0.220 0.046    NA    NA    NA
  55 0.070 0.263 0.411 0.557 0.790 0.834 0.343 0.069    NA    NA
  60 0.035 0.162 0.362 0.644 0.880 1.108 1.155 0.502 0.104    NA
  65 0.004 0.045 0.178 0.481 0.775 1.015 1.314 1.240 0.539 0.122
  70 0.001 0.004 0.041 0.157 0.486 0.682 0.796 1.173 1.203 0.519
  75    NA 0.001 0.003 0.039 0.136 0.342 0.470 0.498 0.955 0.885
  80    NA    NA 0.001 0.001 0.037 0.098 0.158 0.218 0.293 0.536     
\end{verbatim}         
\renewcommand{\baselinestretch}{1.0} 
}
\end{frame} 
\begin{frame} [fragile]
\frametitle{6. Calculation of SMR}
\ \\
We can sum either over individual time slots:
\verb|  > D <- sum(nickLew.ap$d_iap)| \\
\verb|  > E <- sum(nickLew.ap$e_iap)| 

or over the newly formed tables:\\
\verb|  > D <- sum(Obs.lung, na.rm=T)| \\
\verb|  > E <- sum(Exp.lung, na.rm=T)|

\medskip
Either way, the calculation proceeds:
\small
\begin{verbatim}
> SMR <- D/E; SE <- 1/sqrt(D); EF <- exp(1.96*SE)
> round(c(D, E, SMR, SMR/EF, SMR*EF), 2) 
[1] 137.00  26.62   5.15   4.35   6.08
\end{verbatim}
SMR = 5.15 [95\% CI 4.35 to 6.08] \\
$\Rightarrow$ 
substantial excess risk of lung cancer in smelter workers.
\end{frame} 


%\begin{frame}[fragile]
%\frametitle{A model for rates as a function of exposure}
%{\small
%\renewcommand{\baselinestretch}{0.9}  
%\begin{verbatim}  
%> m1 <- glm( Fail ~ expos + I(expos>0) + offset( log(y.iap) ),
%+                 family=poisson, eps=10^-8, data=Nickel.ap )
%
%> round( summary( m1 )$coef, 4 )
%                 Estimate Std. Error  z value Pr(>|z|)
%(Intercept)       -5.2130     0.1543 -33.7839   0.0000
%expos              0.0663     0.0231   2.8670   0.0041
%I(expos > 0)TRUE   0.5570     0.2176   2.5594   0.0105
%\end{verbatim}  
%\renewcommand{\baselinestretch}{1.0} 
%} %$
%\bi
%\item What is this model doing ?
%\item What is the meaning of the parameters?
%\ei
%\end{frame}
%
%\begin{frame}[fragile]
%\frametitle{A model for SMR alone} 
%{\small
%\renewcommand{\baselinestretch}{0.9}  
%\begin{verbatim}  
%> M0 <- glm( Fail ~ 1 + offset( log( e.lung ) ),
%+           family=poisson, eps=10^-8, data=Nickel.ap )
%
%>  round( summary( M0 )$coef, 4 )
%            Estimate Std. Error z value Pr(>|z|)
%(Intercept)   1.6382     0.0854 19.1763        0
%
%> round(exp( ci.lin( M0 )[ ,c(1,5,6)] ), 3)
%Estimate     2.5%    97.5% 
%   5.146    4.353    6.084 
%\end{verbatim}  
%\renewcommand{\baselinestretch}{1.0} 
%}
%
%\end{frame}
%
%\begin{frame}[fragile]
%\frametitle{A model for SMR as a function of exposure}
%{\small
%\renewcommand{\baselinestretch}{0.9}  
%\begin{verbatim}  
%> M1 <- glm( Fail ~ expos + I(expos>0) + offset( log(e.lung) ),
%+                family=poisson, eps=10^-8, data=Nickel.ap)
%> round(cbind(summary(M1)$coef[,1:2], 
%              exp( ci.lin(M1)[ ,c(1,5,6)])),3)
%                 Estimate Std. Error Estimate  2.5% 97.5%
%(Intercept)         1.079      0.154    2.943 2.175 3.982
%expos               0.066      0.023    1.068 1.020 1.118
%I(expos > 0)TRUE    0.673      0.218    1.961 1.278 3.009 
%\end{verbatim}  
%\renewcommand{\baselinestretch}{1.0} 
%} %$
%\bi
%\item What are these models doing ?
%\item What is the meaning of the parameters?
%\ei
%
%\end{frame}


%----------------------------------------------------------------------
\begin{frame} 
\frametitle{Concluding remarks} 

\bi
\item
If specific exposure factors exist that have variable values within
the target cohort, the estimation of rate ratios associated with them
 may be efficiently adjusted for age and calendar period by taking the 
 age- and period-specific expected number as the baseline in Poisson-modelling.
 \medskip
 \item
Follow-up time could be split yet by another relevant time axis, 
like time passed since start of exposure, which may be taken as an explanatory variable when modelling the effects of exposure within a cohort.
\medskip
\item
The main challenge is to identify a sufficiently 
comparable reference population. The ``general'' population is rarely an ideal one. 
\ei
\end{frame}

\end{document}

%----------------------------------------------------------------------
\begin{frame} \frametitle{Cohorts of exposed subjects}
When there is no comparison group we may ask:
Do mortality rates in cohort differ from those of an
\textbf{reference} population?

$\bullet$ Occupational cohorts\\
$\bullet$ Patient cohorts

Compared with reference rates obtained from:

$\bullet$ Population statistics (mortality rates)\\
$\bullet$ Disease registers (hospital discharge registers)\\
$\bullet$ Cancer registers
\end{frame} 

%----------------------------------------------------------------------
\begin{frame} \frametitle{Accounting for age composition}
\begin{itemize}
\item Compare rates in a study group with a standard set of
  age--specific rates
\item Reference rates are normally based on large numbers of cases, so
  they can be assumed to be known
\item Calculate ``expected'' number of cases, $E$, if the standard
  rates had applied in our study group, and compare this with the
  observed number of cases, $D$
\item $\SMR = D/E \quad \sd(\log[\SMR]) = \sqrt{1/D}$
\end{itemize}
\end{frame} 


\end{document}

%----------------------------------------------------------------------
\begin{frame} \frametitle{A likelihood for the SMR}
Statistical model:\\ Cohort rates are proportional to reference rates,
$\lambda_R$:
\[
   \lambda = \theta \times \lambda_R
\]
A Cox model with known baseline hazard rate and $RR=\theta$.
  
$D$ deaths during $Y$ person-years gives the likelihood: 
\bes
  D \ln(\lambda) - \lambda Y
  & = & D \ln(\theta\lambda_R) - \theta\lambda_R Y \\
  & = & D \ln(\theta)+ D \ln(\lambda_R) - \theta(\lambda_R Y) 
\ees 

\newpage
This is (apart from the constant $D \ln(\lambda_R)$) the
likelihood for a rate, $\theta$, based on $D$ deaths during
$\lambda_RY$, so the estimate for $\theta$ is:
\[
  \hat\theta = \frac{D}{\lambda_RY} =
  \frac{\mbox{Observed}}{\mbox{Expected}}
  = \SMR
\]

SMR is the maximum likelihood estimator of the rate ratio between the
cohort and the reference population.

\end{frame} 

%----------------------------------------------------------------------
\begin{frame} \frametitle{Computing person--years}
The person--years computations are carried out by first expanding
records by age band and then by calendar--time bands:

\begin{center}
\setlength{\unitlength}{2pt}
\begin{picture}(210,90)(0,35)
\small
 \put(0,80){\makebox(0,0)[tl]{Age}}
 \put(50,80){\line(1,0){150}}
 \put(50,80){\line(0,1){5}}
 \put(100,80){\line(0,1){5}}
 \put(150,80){\line(0,1){5}}
 \put(200,80){\line(0,1){5}}
 \put(50,77){\makebox(0,0)[t]{35}}
 \put(100,77){\makebox(0,0)[t]{40}}
 \put(150,77){\makebox(0,0)[t]{45}}
 \put(200,77){\makebox(0,0)[t]{50}}

 \put(0,50){\makebox(0,0)[tl]{Year}}
 \put(60,50){\line(1,0){150}}
 \put(60,50){\line(0,1){5}}
 \put(110,50){\line(0,1){5}}
 \put(160,50){\line(0,1){5}}
 \put(210,50){\line(0,1){5}}
 \put(60,47){\makebox(0,0)[t]{1960}}
 \put(110,47){\makebox(0,0)[t]{1965}}
 \put(160,47){\makebox(0,0)[t]{1970}}
 \put(210,47){\makebox(0,0)[t]{1975}}

 \put(0,110){\makebox(0,0)[l]{y}}
 \put(90,110){\line(1,0){90}}
 \put(90,105){\line(0,1){10}}
 \put(100,105){\line(0,1){10}}
 \put(110,105){\line(0,1){10}}
 \put(150,105){\line(0,1){10}}
 \put(160,105){\line(0,1){10}}
 \put(180,105){\line(0,1){10}}
 \put(95,115){\makebox(0,0)[b]{1}}
 \put(105,115){\makebox(0,0)[b]{1}}
 \put(130,115){\makebox(0,0)[b]{4}}
 \put(155,115){\makebox(0,0)[b]{1}}
 \put(170,115){\makebox(0,0)[b]{2}}
\end{picture}
\end{center}


%----------------------------------------------------------------------
\begin{frame} \frametitle{Expected numbers in practice}
\begin{itemize}
\item From the file with the expanded follow-up data:
\begin{itemize}
\item $Y$ --- The risk time in the record
\item age class --- The ageclass of the record
\item period --- The period of the record
\end{itemize}

\item From the file with reference rates:
\begin{itemize}
\item $\lambda_R$ --- The reference rate.
\item age class --- The ageclass of the population rate
\item period --- The period of the population rate
\end{itemize}
\newpage
\item Population rates are matched up to the follow-up data, 
 using the same variable names for age class and period.
\item Expected numbers are computed for each piece of follow-up:
\[
   E = \lambda_R \times Y
\]

\end{itemize}

\end{frame} 

%----------------------------------------------------------------------
\begin{frame} \frametitle{Comparing SMRs}
The model underlying these calculations is the proportional hazards
model, i.e, {\em  for every age-band:}
\[
 \frac{\lambda}{\lambda_{R}} = \theta = \SMR
\]
With SMRs estimated on two groups we assume:
\begin{eqnarray*}
 \frac{\lambda_{1}}{\lambda_{R}} & = & \theta_{1} = \SMR_1 \\
 \frac{\lambda_{2}}{\lambda_{R}} & = & \theta_{2} = \SMR_2
\end{eqnarray*}
So, \textbf{if} models above are correct:
\[
\frac{\SMR_1}{\SMR_2}
\]
is a correct estimate of the rate ratio.

However it introduces an extra assumption --- the proportionality to
the underlying reference rates.

Sometimes necessary and sensible to compare SMRs.
\end{frame} 

%%% Local Variables: 
%%% mode: latex
%%% TeX-master: "slides"
%%% End: 

\end{document}