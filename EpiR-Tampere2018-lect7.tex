\documentclass[12pt]{beamer}

\newcommand{\toggle}[1]{%
\addtocounter{framenumber}{-1}#1 % Comment this out if extra slides should be omitted
}
\newcommand{\toggleafter}[1]{%
#1\addtocounter{framenumber}{-1} % Comment this out if extra slides should be omitted
}

% \usepackage[latin1]{inputenc}
% \usepackage[danish]{babel}
% \usepackage{booktabs,amsmath,amsbsy}

\usepackage[latin1]{inputenc}
\usepackage[english]{babel}
%\usepackage{graphicx,rotating,booktabs,verbatim,amsmath,amsbsy}
\usepackage{graphicx,rotating,booktabs,verbatim,amsmath}
\usepackage{xcolor}

%\usepackage[ae,hyper]{Rd}
%\usepackage{Sweave}
% \SweaveOpts{results=verbatim, keep.source=TRUE, eps=FALSE, 
%           prefix.string=./graph/stroke}
%%%%%%%%%%%%%%%%%%%%%%%%%%%%%%%%%%%%%%%%%

% \setlength{\topmargin}{-2cm}
% \setlength{\textheight}{18cm}
%\newcommand{\E}{\mathrm{E}}
\renewcommand{\P}{\mathrm{P}}
% \newcommand{\next}{\end{frame}\begin{slide}}
% \newcommand{\lopp}{\end{frame} \end{document}}
% \usepackage{soul}
% \usepackage{hyperref}
%\usepackage[display, coloremph, whitebackground, colormath, 
% colorhighlight]{texpower}
%\usepackage[printout]{texpower}
% \usepackage{color}
% \usepackage{pictex}
% \usepackage{fixseminar}
% \usepackage{graphicx}

\newcommand{\Y}{\mathrm{\bf Y}}
\newcommand{\X}{\mathrm{\bf X}}
\newcommand{\I}{\mathrm{\bf I}}
\newcommand{\y}{\mathrm{\bf y}}
\newcommand{\h}{\mathrm{\bf h}}
\newcommand{\z}{\mathrm{\bf z}}
\newcommand{\x}{\mathrm{\bf x}}
\newcommand{\w}{\mathrm{\bf w}}
\newcommand{\e}{\mathrm{\bf e}}
\renewcommand{\v}{\mathrm{\bf v}}
\newcommand{\bnull}{\mathrm{\bf 0}}
\newcommand{\K}{\mathrm{\bf K}}
\newcommand{\Z}{\mathrm{\bf Z}}
\newcommand{\B}{\mathrm{\bf B}}
\newcommand{\eeta}{\mathrm{\bf \eta}}
\newcommand{\biota}{\mathrm{\bf \iota}}
\newcommand{\bepsilon}{\mathrm{\bf \epsilon}}
\newcommand{\bnu}{\mathrm{\bf \nu}}
\newcommand{\bbeta}{\mathrm{\bf \beta}}
\newcommand{\btheta}{\mathrm{\bf \theta}}
\newcommand{\bzeta}{\mathrm{\bf \zeta}}
\newcommand{\blambda}{\mathrm{\bf \lambda}}
\newcommand{\bLambda}{\mathrm{\bf \Lambda}}
\newcommand{\bGamma}{\mathrm{\bf \Gamma}}
\newcommand{\bPsi}{\mathrm{\bf \Psi}}
\newcommand{\bPi}{\mathrm{\bf \Pi}}

%\definecolor{roheline}{rgb}{0.22,0.49,0.18}
%\definecolor{droheline}{rgb}{0.41,0.76,0.36}

\definecolor{sinine}{rgb}{0.00,0.25,0.50}
\definecolor{dsinine}{rgb}{0.53,0.58,0.73}
\definecolor{punane}{rgb}{0.72,0.05,0.00}
\definecolor{roheline}{rgb}{0.2,0.4,0.1}
\definecolor{heleroheline}{rgb}{0.38,0.75,0.00}
\definecolor{dheleroheline}{rgb}{0.45,1.00,0.45}
\definecolor{droheline}{rgb}{0.8,0.9,0.6}

% \addTPcolor{roheline} \addTPcolor{sinine} \addTPcolor{heleroheline}
% \def\naide#1{\color{roheline}  #1}
% How visible should the uncovered items be? 0 corresponds to not at all.
% typeset using LaTeX2e
%\documentclass[12pt]{beamer}
% \documentclass[12pt]{beamer}
\mode<presentation>
{
% \usetheme{Singapore}
% \setbeamercovered{transparent}
}
\usepackage[latin1]{inputenc}
\usepackage[english]{babel}
\usepackage{epsfig}
\usepackage{rotating}
\usepackage{graphicx}
%\usepackage{mslapa}
\usepackage{amsmath}
%\usepackage[all]{xy}
% \usepackage{amssymb,graphicx}
%\input psfig.sty
\usepackage{booktabs}
\usepackage{graphpap}
\usepackage{verbatim}
%\usepackage{amsbsy}
\usepackage{caption}
\captionsetup{labelformat=empty,labelsep=none}
\usepackage{multicol}

\setbeamertemplate{footline}[page number]
\setbeamertemplate{navigation symbols}{}
\DeclareGraphicsExtensions{.pdf, .jpg}
% \DeclareGraphicsExtensions{.eps}

\newcommand{\Rlogo}[1]{\includegraphics[#1]{Rlogo}}

 \input{usefulesa}
\definecolor{oy_darkblue}{RGB}{0,55,90}
% \newcommand{\R}{\bf \textsf{R}}
\parskip\medskipamount



\title{Epidemiologic Data Analysis using R\\
Part 7: Analysis of survival data}  % : \\ Analysis of Follow-up Studies}
% \Rlogo{height=1em} }

\author{Janne Pitk{\"a}niemi \\
{\scriptsize Esa L{\"a}{\"a}r{\"a}}}

\institute{Finnish Cancer Registry, Finland,
 \texttt{<janne.pitkaniemi@cancer.fi>} \newline
{\scriptsize University of Oulu, Finland,  
 \texttt{<esa.laara@oulu.fi>}}}


\date{Tampere School of Health Sciences, % University of Tampere/Postgraduate training,
Jan 17-Feb 28,  2014}

%\begin{center} \Rlogo{height=2em} \end{center}

\AtBeginSubsection[]
{
  \begin{frame}<beamer>
    \frametitle{Outline}
    \tableofcontents[currentsection,currentsubsection]
  \end{frame}
}

%\beamerdefaultoverlayspecification{<+->}

\begin{document}

\definecolor{darkgreen}{rgb}{0,.5,0}

\begin{frame}
    \titlepage
\end{frame}


\begin{frame}
\frametitle{Topics somewhat covered}

\begin{itemize}
\item[1.] Survival or time to event data \& censoring.
 \medskip
\item[2.] 
 Probability concepts for times to event: \\ 
 survival, hazard and cumulative hazard functions,
 \medskip
\item[3.] Kaplan--Meier and Nelson--Aalen estimators.
 \medskip
 \item[4.] 
 Regression modelling of hazards: Cox model.
 \medskip
 \item[5.]
 Packages \texttt{survival}.
\medskip
\item[6.] 
 Functions \texttt{Surv(), survfit(), survMisc(), coxph()}.
\end{itemize}

\end{frame}


\begin{frame}
\frametitle{Survival time -- time to event}

Let $T$ be the \textbf{time} spent in a given \textbf{state} from its 
beginning till a certain \textit{endpoint} or \textit{outcome} \textbf{event} or \textit{transition}
 occurs, changing the state to another. \\
 (\texttt{lex.Cst - lex.dur - lex.Xst})

\bigskip
Examples of such times and outcome events:
\begin{itemize}
\item lifetime: birth $\rightarrow$ death,
\medskip
\item duration of marriage: wedding $\to$ divorce, 
\medskip
\item healthy exposure time: \\ start of exposure  
  $\rightarrow$ onset of disease,
  \medskip
\item clinical survival time: \\
 diagnosis of a disease  $\rightarrow$ death.
\end{itemize}
\end{frame}

\begin{frame}
   \frametitle{Set-up of classical survival analysis} 

\begin{itemize}
\item
\textbf{Two-state model}: only one type of event changes the initial state.
\medskip
\item
Major applications: analysis of lifetimes
 since birth and of survival times since diagnosis of a disease 
 until death from any cause.
\end{itemize}

\setlength{\unitlength}{0.7pt}
% \begin{center}
\begin{picture}(400,80)(-40,70)
  \thicklines
  \put(  0, 80){\framebox(110,50){Alive}}
  \put(240, 80){\framebox(110,50){Dead}}
  \put(125,105){\vector(1, 0){100}}
  \put(170,110){\makebox(0,0)[b]{Transition}}
\end{picture}
% \end{center}
 \textbf{Censoring}: Death and final lifetime not observed
  for some subjects 
  %, as the follow-up terminates 
  due to emigration or closing the follow-up while they are still
 alive 

\end{frame}


\begin{frame}
\frametitle{Distribution concepts: survival function} 

Cumulative distribution function (CDF)
$F(t)$ and density function $f(t) =F'(t)$ of survival time $T$:
\[F(t) =  P( T \le t) = \int_0^t f(u) du \]
= \textbf{risk} or probability that the event occurs by $t$. 
% (probability of surviving at most up to $t$).

\bigskip
\textbf{Survival function} 
%\textcolor{red}{
\[ S(t) = 1- F(t) =  P( T  >  t) = \int_t^{\infty} f(u)du, \]
= probability of avoiding the event at least up to $t$
$\qquad\qquad{}$ (the event occurs
only after $t$).  

\end{frame}



\begin{frame}

\frametitle{Distribution concepts: hazard function}

The \textbf{hazard rate} or \textbf{intensity} function $\lambda(t)$
\begin{align*}
\lambda(t) & = \underset{\Delta \rightarrow 0}{\lim} 
 {P(t < T \le t+\Delta | T > t)}/{\Delta} \\
   & = \underset{\Delta \rightarrow 0}{\lim} 
      \frac{P(t < T \le t+\Delta)/\Delta }{P(T > t)}
     = \frac{f(t)}{S(t)}  
\end{align*}
\begin{itemize}
\item[$\approx$]  the conditional probability that
the event occurs in a short
 interval $(t, t+\Delta]$, given that it does not
occur before $t$, per interval length. 
\end{itemize}

In other words, during a short interval
 \begin{center}
 risk of event $\approx$ hazard $\times$ interval length 
 \end{center}

\end{frame}

\begin{frame}
\frametitle{Distribution: cumulative hazard etc.}

The \textbf{cumulative hazard} (or integrated intensity):
\[
\Lambda(t) = \int_0^t \lambda(v)dv
\]

\end{frame}


\begin{frame}
\frametitle{Observed data on survival times}

For individuals $i = 1, \dots, n$ let \\
% $\quad B_i$ = time of entry to follow-up (often $B_i = 0$),  \\
$\quad T_i$ = true time to event,\\
% $\quad C_i$ = variable for event 1, or 2, or censoring ,\\
$\quad U_i$ = true time to censoring.\\
 Censoring is assumed \textbf{noninformative}, \textit{i.e.} independent 
 from occurrence of events.
 
We observe 
\begin{itemize}
\item[ ]
$y_i = \text{min}\{ T_i, U_i \}$, \textit{i.e.}
the exit time, and
\item[ ]
 $ \delta_{i} = 1_{ \{ T_i < U_i  \} }$, 
  indicator (1/0) for the event occurring first, before censoring. 
\end{itemize}

Censoring must properly be taken into account in the statistical analysis.

% both in parametric likelihood-based inference and in non-parametric approaches.

\end{frame}


\begin{frame}
\frametitle{Approaches for analysing survival time}

\begin{itemize}
\item 
\textbf{Parametric models} on hazard rate $h(t)$ \\ % fitted on survival times
(like % exponential, %: $h(t) = \lambda$ (constant over time)
Weibull, gamma, etc.) %: $h(t) = \kappa \lambda t^{\kappa-1}$
 %: $f(t) =  \lambda^\alpha 
%        t^{\alpha - 1} \exp( - \lambda t ) / \Gamma(\alpha)$
% lognormal, 
% $$ f(t) = \frac{1}{ y \sqrt{2\pi\sigma^2} }
%   \exp \left\{ - \frac{ [\log(t) - \mu]^2 }{2\sigma^2}  \right\}  $$
% log-logistic, generalized gamma, \textit{etc.}) \\ 
 -- Likelihood:
\begin{align*} L & = \prod_{i=1}^n \lambda(y_i)^{\delta_i} S(y_i) \\
  & = 
   \exp\left\{ \sum_{i=1}^n 
     [ \delta_i \: \log \: \lambda(y_i) - \Lambda(y_i) ] 
       \right\} 
\end{align*}   
\item 
\textbf{Piecewise constant rate} model on $\lambda(t)$ \\ 
-- estimation of $\hat\lambda(t)$ with poisson regression 
\item 
\textbf{Non-parametric} methods, 
like \\ Kaplan--Meier (KM) % and Aalen--Johansen (AJ) 
estimator of survival curve $S(t)$ and Cox % and Fine \& Gray 
proportional hazards model.
% \\ -- The focus in this presentation.

\end{itemize}
\end{frame}

\begin{frame}[fragile]

\frametitle{R package \texttt{survival} }

Tools for analysis with one outcome event.

\begin{itemize}
\item 
\texttt{Surv(time,event) -> sobj} \\ 
creates a \textbf{survival object} \texttt{sobj}, % from follow-up time 
  % (or from \textit{entry} and \textit{exit} times) and event  indicator,
\item 
\texttt{survfit(sobj) -> sfo} \\
 non-parametric survival curve estimates, like KM 
% from survival object  \texttt{sobj} 
(also estimated baseline in a Cox model), 
\item 
\texttt{plot(sfo)} \\
% applied to \texttt{survfit} object \texttt{sfo}
 survival curves and related graphs, 
\item 
\verb|coxph(sobj ~ x1 + x2 +...)| \\ 
fits the Cox model
% for the relative hazards to depend 
with covariates \texttt{x1} and \texttt{x2}. 
\item 
\texttt{survreg()} -- parametric survival models.
\end{itemize}   
\end{frame}

\begin{frame}
\frametitle{KM esimate for survival function $S(t)$}

\begin{itemize}
\item  Order event times (possibly separately in groups)
\item $\hat{S(t_j)}=\hat{S(t_{j-1})} (1-\frac{d_j}{n_j}) $, where $t_0=0$ and
${\hat{S(0)}=1}$
\item ${\hat{S(t)}}$ is constant between event -- step function 
\item Each observations contributes as long as at risk for the event and confidence intervals can be introduced using classical inference framework.

\end{itemize}   
\end{frame}

\begin{frame}
\frametitle{KM esimate for survival function $S(t)$}

\begin{itemize}
\item  Order event times (possibly separately in groups)
\item $\hat{S(t_j)}=\hat{S(t_{j-1})} (1-\frac{d_j}{n_j}) $, where $t_0=0$ and
${\hat{S(0)}=1}$
\item ${\hat{S(t)}}$ is constant between event -- step function 
\item Each observations contributes as long as at risk for the event and confidence intervals can be introduced using classical inference framework.

\end{itemize}   
\end{frame}

\begin{frame}[fragile]
\frametitle{Veterans' Administration Lung Cancer study}

In this trial, males with advanced inoperable lung cancer were randomized to a standard therapy and a test chemotherapy. The primary endpoint for the therapy comparison was time to death in days, represented by the variable Time.

Variables 

\begin{itemize}
\item trt:	 1=standard 2=test
\item celltype:1=squamous, 2=smallcell, 3=adeno, 4=large
\item time:	 survival time (days)
\item status:	 status 1= death, 0=censored 
\item karno:	 Karnofsky performance score (100=good)
\item diagtime:months from diagnosis to randomisation
\item age:	 in years
\item prior:	 prior therapy 0=no, 1=yes
\end{itemize}

{\scriptsize Reference: D Kalbfleisch and RL Prentice (1980), The Statistical Analysis of Failure Time Data. Wiley, New York.}
\end{frame}


\begin{frame}[fragile]
\frametitle{Veteran data}

{\scriptsize \begin{verbatim}
> head(veteran)
  trt celltype time status karno diagtime age prior
1   1 squamous   72      1    60        7  69     0
2   1 squamous  411      1    70        5  64    10
3   1 squamous  228      1    60        3  38     0
4   1 squamous  126      1    60        9  63    10
5   1 squamous  118      1    70       11  65    10
6   1 squamous   10      1    20        5  49     0   
...
70    2  squamous  999      1    90       12  54    10
71    2  squamous  112      1    80        6  60     0
72    2  squamous   87      0    80        3  48     0
73    2  squamous  231      0    50        8  52    10
74    2  squamous  242      1    50        1  70     0

\end{verbatim}
}
\end{frame}


\begin{frame}[fragile]
\frametitle{Estimate for hazard function $\lambda(t)$ in R \newline -- splitting time scale}

{\scriptsize 
\begin{verbatim}
#===============================
# get packages needed for the analysis
#===============================
library(dlpyr)
library(survival)
library(ggplot2)
library(Epi)
library(dummies)
#===============================
# define stating time at 0 and create subject id
#===============================
veteran$start<-0
veteran$id<-1:nrow(veteran)
#===============================
# Split follow-up time into interval
#===============================
nvet<-survSplit(
veteran,
cut=c(100,200,300,400),
event="status",
start="start",
end="time",episode="period")
\end{verbatim}
}
\end{frame}

\begin{frame}[fragile]
\frametitle{Estimate for hazard function $\lambda(t)$ in R \newline
 -- splitting time scale}
{\scriptsize \begin{verbatim}
#===============================
# dplyr -package,rather than stat.table() -function 
#===============================
gsvet<-group_by(nvet,period)
speriod<-summarize(gsvet, 
               n=length(trt),
               events=sum(status),
               pyrs=round(sum(time-start)),
               rate1000=(events/pyrs)*1000,
               lograte=log(events/pyrs))
speriod<-data.frame(speriod)
speriod<-cbind(speriod,dummy(speriod$period))
speriod

> speriod[,1:6]
  period   n events pyrs rate1000   lograte
1      0 137     79 8692 9.088817 -4.700710
2      1  53     26 3502 7.424329 -4.902993
3      2  24     10 1807 5.534034 -5.196838
4      3  13      7 1054 6.641366 -5.014438
5      4   6      6 1608 3.731343 -5.590987
\end{verbatim}
}
\end{frame}

\begin{frame}[fragile]
\frametitle{Estimate for hazard function $\lambda(t)$ in R \newline
-- Poisson regression}

{\scriptsize 
\begin{verbatim}
#===============================
# Summarize data with data.table, rather than stat.table() -function 
#===============================
gsvet<-group_by(svet,period)
speriod<-summarize(gsvet, 
               n=length(trt), events=sum(status),
               pyrs=round(sum(time-start)), rate1000=(events/pyrs)*1000,
               lograte=log(events/pyrs))
speriod<-data.frame(speriod)
speriod<-cbind(speriod,dummy(speriod$period))
#===============================
#  Poisson model for incidence in each period of time
#===============================
m<-glm(events~1-1+speriod0+speriod1+speriod2+speriod3+speriod4,
       family=poisson,offset=log(pyrs),data=speriod)
summary(m)
round(ci.lin(m,Exp=TRUE)[,-(3:4)],6)
          Estimate   StdErr exp(Est.)     2.5%    97.5%
speriod0 -4.700710 0.112509  0.009089 0.007290 0.011331
speriod1 -4.902993 0.196116  0.007424 0.005055 0.010904
speriod2 -5.196838 0.316228  0.005534 0.002978 0.010285
speriod3 -5.014438 0.377964  0.006641 0.003166 0.013931
speriod4 -5.590987 0.408248  0.003731 0.001676 0.008306
\end{verbatim}
}
\end{frame}

\begin{frame}[fragile]
\frametitle{Estimation of the survival function S(t)}

{\scriptsize \begin{verbatim}
Call: survfit(formula = Surv(time, status) ~ trt, data = veteran)
                trt=1 
         n        d
 time n.risk n.event survival std.err lower 95% CI upper 95% CI
    3     69       1   0.9855  0.0144      0.95771        1.000
    4     68       1   0.9710  0.0202      0.93223        1.000
    7     67       1   0.9565  0.0246      0.90959        1.000
    8     66       2   0.9275  0.0312      0.86834        0.991
   10     64       2   0.8986  0.0363      0.83006        0.973
   
   S(0)=1; S(3)=S(0)*(1-(1/69) )=0.985 ;S(4)=S(3)*(1-(1/68)) =0.971

                trt=2 
 time n.risk n.event survival std.err lower 95% CI upper 95% CI
    1     68       2   0.9706  0.0205      0.93125        1.000
    2     66       1   0.9559  0.0249      0.90830        1.000
    7     65       2   0.9265  0.0317      0.86647        0.991
    8     63       2   0.8971  0.0369      0.82766        0.972
   13     61       1   0.8824  0.0391      0.80900        0.962   
   
  S(0)=1; S(1)=S(0)*(1-(2/68))   =0.971; S(2)=S(1)*(1-(1/66)) =0.956
   
\end{verbatim}
}
\end{frame}


\begin{frame}[fragile]
\frametitle{Plot of Survival curve for Veteran data}

{\scriptsize \begin{verbatim}
library(survival)
library('SurvMisc')
m<-survfit(Surv(time, status)~trt, data=veteran)
summary(m)
autoplot(m,CI=TRUE,bands=TRUE,divideTime=100,plotTable=TRUE)
\end{verbatim}
}
\begin{figure}
\centering
\label{fig:KMVeteran}
\includegraphics[width=1.0\linewidth]{E:/figures/KMVeteran}
\end{figure}
\end{frame}

\begin{frame}[fragile]
\frametitle{Estimate cumulative hazard function -- $\hat \Lambda(t)$ }

\begin{itemize}
\item KM for survival function ($P(T>t)$) is often presented
\item Estimation of the cumulative hazard enables risk calculations and can be applied to competing risk situation (cancer and mortality aer competing -- do not use naive KM)
\item Cumulative rate - time(age) band weighted sum of hazards --
estimate of $\hat \Lambda(t)$
\item Risk: $P(T<t) =F(t)=1-S(t)=1-exp(- \Lambda(t))$
\item If low incidence or short risk period:\newline  $1-exp(- \Lambda(t)) \approx \Lambda(t)  $ \quad i.e. rate $\times$ period at risk
\item Cumulative hazard can be estimated from KM, but Nelson-Aalen or
Nelson-Johansen should be preferred
\end{itemize}
\end{frame}

\begin{frame}[fragile]
\frametitle{Estimate cumulative hazard function -- $\hat \Lambda(t)$}

{\scriptsize \begin{verbatim}
m1<-calcSurv(Surv(veteran$time[veteran$trt==1],veteran$status[veteran$trt==1]))
m2<-calcSurv(Surv(veteran$time[veteran$trt==2],veteran$status[veteran$trt==2]))
head(m1)
head(m2)

   t  n e       SKM      SKMV      HNel     SNelA     HNelV       HKM
1  3 69 1 0.9855072 0.0002070 0.0144928 0.9856118 0.0002100 0.0145988
2  4 68 1 0.9710145 0.0004079 0.0291986 0.9712235 0.0004263 0.0294139
3  7 67 1 0.9565217 0.0006027 0.0441240 0.9568353 0.0006491 0.0444518
4  8 66 2 0.9275362 0.0009741 0.0744270 0.9282752 0.0011082 0.0752234
5 10 64 2 0.8985507 0.0013211 0.1056770 0.8997152 0.0015965 0.1069721
6 11 62 1 0.8840580 0.0014855 0.1218061 0.8853200 0.0018566 0.1232326
...
t time
n no. at risk
e no. events
KM Survival estimate by Kaplan-Meier (Product-Limit) estimator
KMV	 Variance of Kaplan-Meier estimate (Greenwoods formula)
SNelA Survival estimate from Nelson-Aalen estimator: S=exp(H)
HNel Nelson-Aalen estimate of cumulative hazard function
HNelV Variance of Nelson-Aalen estimate
HKM	Cumulative hazard estimate from Kaplan-Meier estimator: H = -log(S)
\end{verbatim}
}
\end{frame}

\begin{frame}

\frametitle{Regression models for time-to-event data}

Consider only one outcome \& no competing events
 \begin{itemize} 
 \item
Subject $i$ $(i=1, \dots, n)$ has an own vector $x_i$ that contains
 values $(x_{i1}, \dots, x_{ip})$ of a set of $p$ 
continuous and/or binary covariate terms.
%\pause
\medskip
\item
 In the spirit of generalized linear models we let 
  $\beta = (\beta_1, \dots, \beta_p)$ be regression coefficients
  and build  a \textbf{linear predictor} 
 $$ \eta_i =
x_i^{\small\text{T}} \beta =  \beta_1 x_{i1} + \dots + \beta_p x_{ip} $$ 
\item
Specification of outcome variable? \\
 Distribution (family)? Expectation? Link?  
\end{itemize} 
\end{frame}


\begin{frame}
\frametitle{Regression models (cont'd)}
Survival regression models can be defined \textit{e.g.} for 
\begin{itemize}
\item[(a)] survival times directly 
$$ \log(T_i) = \eta_i + \epsilon_i, \quad\text{s.t. } 
\epsilon_i \sim F_0(t; \alpha)$$ 
where $F_0(t; \alpha)$ is some baseline model, 
\medskip
%\pause
\item[(b)] hazards, multiplicatively: $$ 
\lambda_i(t) = \lambda_0(t; \alpha) r(\eta_i), \quad\text{where}$$
$\lambda_0(t; \alpha)$ = baseline hazard and \\
$r(\eta_i)$ = relative rate function, typically $\exp(\eta_i)$
\medskip
%\pause
\item[(c)] hazards, additively: 
$$ \lambda_i(t) = \lambda_0(t; \alpha) + \eta_i. $$
\end{itemize}
\end{frame}




\begin{frame}
\frametitle{Relative hazards model or Cox model}

In model (b), the baseline hazard $\lambda_0(t,\alpha)$ may be given a parametric form (\textit{e.g.} Weibull) or
a piecewise constant rate (exponential) structure.

\bigskip
Often a parameter-free form $\lambda_0(t)$ is assumed. Then
\[
  \lambda_i(t) = \lambda_0(t) \exp(\eta_1),
\]
specifies the \textbf{Cox model} or the \textbf{semiparametric proportional hazards model}.

\bigskip
$\eta_i = \beta_1 x_{i1} + \dots + \beta_p x_{ip}$ not depending on time.  

\bigskip
Generalizations: \textbf{time-dependent} \\ covariates $x_{ij}(t)$, and/or 
effects $\beta_j(t)$.

% For exponential model, $\lambda_0(t) = \lambda_0$.

% Piecewise exponential model $\to$ constant baseline
% hazard in successive intervals:   
% \[ 
% \lambda_0(t) = \lambda_{0k}, \mbox{ for } 
% a_{k-1}} < t \leq a_k, \quad k = 1, 2, \dots, K 
% \]

\end{frame}

\begin{frame}
\frametitle{PH model: interpretation of parameters}

Present the model explicitly in terms of $x$'s and $\beta$'s.
\[
\lambda_i(t) = \lambda_0(t)  \exp({\beta_1 x_{i1} + \dots +
\beta_p x_{ip}})
\]
Consider two individuals, $i$ and $i'$, having the same values of all
other covariates except the $j^{\text{th}}$ one.

\bigskip
The ratio of hazards is constant:
$$  \frac{\lambda_i(t)}{\lambda_{i'}(t)} = \frac{\exp( \eta_{i}) }{\exp(\eta_{i'})}
= \exp \{ \beta_j(x_{ij}-x_{i'j}) \} . $$
Thus $e^{\beta_j} = \text{HR}_j$ = \textbf{hazard ratio} or relative rate
 associated with
 a unit change in covariate $X_j$.

\end{frame}



\begin{frame}[fragile]
\frametitle{Ex. Veteran data and treatment effect}

Fitting Cox models with trt effect.
\scriptsize
\begin{verbatim}
Call:
coxph(formula = Surv(time, status) ~ trt, data = veteran)

  n= 137, number of events= 128 

       coef exp(coef) se(coef)     z Pr(>|z|)
trt 0.01774   1.01790  0.18066 0.098    0.922

    exp(coef) exp(-coef) lower .95 upper .95
trt     1.018     0.9824    0.7144      1.45

Concordance= 0.525  (se = 0.026 )
Rsquare= 0   (max possible= 0.999 )
Likelihood ratio test= 0.01  on 1 df,   p=0.9218
Wald test            = 0.01  on 1 df,   p=0.9218
Score (logrank) test = 0.01  on 1 df,   p=0.9218
\end{verbatim}
\normalsize
HR for treatment 2 vs. 1 is 1.02 (95\% CI 0.71;1.45) \newline
Not statistically significant (p=0.92)
\end{frame}



\begin{frame}
\frametitle{Proportionalilty of hazards?}
\begin{itemize}
% \item
%\emph{How to check this key assumption?}
\item
Consider two groups $g$ and $h$ defined by one categorical covariate, and let $\rho > 0$. 

\medskip
If $\lambda_g(t) = \rho \lambda_h(t)$  then 
$\Lambda_g (t) = \rho \Lambda_h(t)$ and
$$ \log\: \Lambda_g  (t)  = \log (\rho) + \log\: \Lambda_h (t), $$
thus log-cumulative hazards should be parallel!
\bigskip
\item[$\Rightarrow$] 
\textit{Plot the estimated log-cumulative hazards and see
whether they are sufficiently parallel}.
\bigskip
\item
% Recall that $\log\: \Lambda (t)  = \log[- \log \: S(t)]$. Hence,
\texttt{plot(coxobj, ..., fun = 'cloglog')}
% meaning \textbf{complementary log-log} transform of $S(t)$,
% produces the desired graph.
\medskip
\item
Testing the proportionality assumptions: \texttt{cox.zph(coxobj)}.
\end{itemize}
\end{frame}

\begin{frame}[fragile] 
\frametitle{Ex. Veteran data - test PH}
\begin{itemize}
\item With $>1$ covariates,  \texttt{cox.zph()} tests the assumption by checking, whether the corresponding parameters (\& hazard ratios)
may vary in time.

\item Suppose that I want to include information on patient baseline general disease status -- Karnofsky performance score (0=dead, 100=good)

\item Dichotomize the Karnofsky score 0 if Score $[$0,50 $]$ and 1 if $($50,100$]$
\end{itemize}
\end{frame}

\begin{frame}[fragile] 
\frametitle{Ex. Veteran data - test PH}
{\scriptsize 
\begin{verbatim}
> veteran$karnod<-as.numeric(veteran$karno>50)
> m<-coxph(formula = Surv(time, status) ~ trt+ karnod, data = veteran)
> summary(m)
  n= 137, number of events= 128 

          coef exp(coef) se(coef)      z Pr(>|z|)    
trt     0.1176    1.1248   0.1826  0.644    0.519    
karnod -0.9790    0.3757   0.1895 -5.165  2.4e-07 ***

       exp(coef) exp(-coef) lower .95 upper .95
trt       1.1248      0.889    0.7864    1.6088
karnod    0.3757      2.662    0.2591    0.5447

#testing proportionality
> cox.zph(m)
          rho chisq        p
trt    -0.152   3.1 7.85e-02
karnod  0.365  15.2 9.82e-05
GLOBAL     NA  16.7 2.33e-04
\end{verbatim}
}
Test for proportionality are significant (p$<$0.05) -- assumptions of proportionalty of hazards is rejected for both tretment and score variable
-- stratify accoding to the score 
\end{frame}

\begin{frame}[fragile] 
\frametitle{Ex. Veteran data - test PH}
{\scriptsize 
\begin{verbatim}
> m<-coxph(formula = Surv(time, status) ~ trt+strata(karnod), 
+          data = veteran)
> summary(m)
Call:
coxph(formula = Surv(time, status) ~ trt + strata(karnod), data = veteran)

  n= 137, number of events= 128 

       coef exp(coef) se(coef)     z Pr(>|z|)
trt 0.01351   1.01360  0.18372 0.074    0.941

    exp(coef) exp(-coef) lower .95 upper .95
trt     1.014     0.9866    0.7071     1.453

> cox.zph(m)
       rho chisq     p
trt -0.133  2.31 0.129
\end{verbatim}
}
Test for proportionality is not significant (p$>$0.05) \newline
HR for treatment effect is 1.01 95\% CI (0.71 ;1.45)
\end{frame}


\begin{frame}[fragile]
\frametitle{Homogeneity of HRs}

Question: Are the HRs for different celltypes similar or not $?$ 

Testing hypothesis of regression coefficients equal (Not the hypothesis that they are zero)

More formally, the model: 
\[
\lambda_i(t) = \lambda_0(t)  \exp({\beta_1 x_{i1} + \beta_2 x_{i2} + \beta_3 x_{i3} })
\]
, where $x_{i1}=1$ if celltype smallcell 0 otherwise \newline
and $x_{i2}=1$ if celltype adeno 0 otherwise \newline
and $x_{i3}=1$ if celltype large 0 otherwise \newline
and squamous celltype has been chosen as the reference category
($x_{i1}=0$ and $x_{i2}=0$ and $x_{i3}=0$)


\end{frame}


\begin{frame}[fragile]
\frametitle{Homogeneity of HRs}


Testing hypothesis of regression coefficients equal (Not the hypothesis that they are zero)

Hypothesis more formally: 
$$
H_0: \beta_1=\beta_2=\beta_3 $$
$$
H_A: \beta_1 \ne \beta_2 \  and \ \beta_2 \ne \beta_3 
$$

Note that this test is different from testing hypothesis 
$$
H_0: \beta_1=0 \ and \ \beta_2=0 \ and \  \beta_3=0
$$

\end{frame}

\begin{frame}[fragile]
\frametitle{Homogeneity of HRs}

{\scriptsize \begin{verbatim}
> m<-coxph(formula = Surv(time, status) ~ celltype, 
+          data = veteran)
> summary(m)
Call:
coxph(formula = Surv(time, status) ~ celltype, data = veteran)

  n= 137, number of events= 128 

                    coef exp(coef) se(coef)     z Pr(>|z|)    
celltypesmallcell 1.0013    2.7217   0.2535 3.950 7.83e-05 ***
celltypeadeno     1.1477    3.1510   0.2929 3.919 8.90e-05 ***
celltypelarge     0.2301    1.2588   0.2773 0.830    0.407    
---

                  exp(coef) exp(-coef) lower .95 upper .95
celltypesmallcell     2.722     0.3674     1.656     4.473
celltypeadeno         3.151     0.3174     1.775     5.594
celltypelarge         1.259     0.7944     0.731     2.168

Concordance= 0.608  (se = 0.029 )
Rsquare= 0.166   (max possible= 0.999 )
Likelihood ratio test= 24.85  on 3 df,   p=1.661e-05
Wald test            = 24.09  on 3 df,   p=2.387e-05
Score (logrank) test = 25.51  on 3 df,   p=1.208e-05
\end{verbatim}
}
\end{frame}

\begin{frame}[fragile]
\frametitle{Homogeneity of HRs}

Let's look first are the HRs for celltypes pairwisely similar
{\scriptsize 
\begin{verbatim}
> linearHypothesis(m,c("celltypesmallcell=celltypeadeno"))
Linear hypothesis test
Hypothesis: celltypesmallcell - celltypeadeno = 0

Model 1: restricted model
Model 2: Surv(time, status) ~ celltype

  Res.Df Df  Chisq Pr(>Chisq)
1    135                     
2    134  1 0.3452     0.5569
> linearHypothesis(m,c("celltypeadeno=celltypelarge"))
Linear hypothesis test
Hypothesis: celltypeadeno - celltypelarge = 0

Model 1: restricted model
Model 2: Surv(time, status) ~ celltype

  Res.Df Df  Chisq Pr(>Chisq)   
1    135                        
2    134  1 10.153   0.001441 **
\end{verbatim}
}
The HRs for Adeno and large celltype cancers are not significantly (p=0.56),  but adeno and large type are when compred to squamous celltype.  

\end{frame}


\begin{frame}[fragile]
\frametitle{Homogeneity of HRs}

Joint the for homogeneity of HRs

{\footnotesize \begin{verbatim}
> linearHypothesis(m,c("celltypesmallcell=celltypeadeno",
+                      "celltypeadeno=celltypelarge")
+                  )
Linear hypothesis test

Hypothesis:
celltypesmallcell - celltypeadeno = 0
celltypeadeno - celltypelarge = 0

  Res.Df Df  Chisq Pr(>Chisq)   
1    136                        
2    134  2 12.328   0.002104 **

\end{verbatim}
}
HR's are not the same between celltypes

\end{frame}


\begin{frame}[fragile]
\frametitle{Predictions from the Cox model}
\begin{itemize}
\item
Individual survival \textit{times} cannot be predicted
but ind'l survival \emph{curves} can.
PH model implies:
\[
S_i(t) = [S_0(t) ]^{\exp(\beta_1 x_{i1} +\ldots+\beta_p x_{ip})}
\]
\item
Having  estimated $\beta$ by partial likelihood, 
the baseline $S_0(t)$ is estimated by Breslow method 
\item 
\medskip
 From these, a survival curve for an individual
with given covariate values is predicted.
\item
\medskip
In R: 
\texttt{pred <- survfit(mod, newdata=...)} 
and \texttt{plot(pred)}, where \texttt{mod} is the fitted
\texttt{coxph} object, 
and \texttt{newdata}  
specifies the covariate values.
\end{itemize}
\end{frame}


\begin{frame}
\frametitle{Modelling with competing risks}

% When more than one outcome are operating,
% one may consider the 
Main options, providing answers to different questions.

\begin{itemize}
\item[(a)]
  Cox model for event-specific hazards $h_c(t) = f_c(t)/[1-F(t)]$, when \textit{e.g.} the interest is in the biological effect of the prognostic factors on the fatality of the very disease that often leads to the relevant outcome.  
  \bigskip 
\item[(b)]
 \textbf{Fine--Gray model} for the hazard  of the subdistribution $\gamma_c(t) = f_c(t)/[1-F_c(t)]$ 
  when we want to assess the impact of the factors on the overall cumulative incidence of event $c$.  \\
  -- Function \texttt{crr()} in package \texttt{cmprsk}. 
\end{itemize}
\end{frame}


\end{document}