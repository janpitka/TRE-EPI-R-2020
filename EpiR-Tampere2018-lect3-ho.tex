% typeset using LaTeX2e
\documentclass[handout, 12pt]{beamer}
% \documentclass[12pt]{beamer}
%\mode<presentation>
%{
% \usetheme{Singapore}
% \setbeamercovered{transparent}
%}
% making handouts
%\usepackage[overlay]{textpos}
%\usepackage{pgfpages}
%\pgfpagesuselayout{4 on 1}[letterpaper, border shrink = 5mm]
%\mode<presentation>{
%  \setbeamercovered{invisible}
%}
% end of making handouts

\usepackage{pgfpages}
\pgfpagesuselayout{4 on 1}[a4paper,landscape,border shrink=5mm]

\usepackage[latin1]{inputenc}
\usepackage[english]{babel}
\usepackage{epsfig}
\usepackage{rotating}
\usepackage{graphicx}
%\usepackage{mslapa}
\usepackage{amsmath}
%\usepackage[all]{xy}
% \usepackage{amssymb,graphicx}
%\input psfig.sty
\usepackage{booktabs}
\usepackage{graphpap}
\usepackage{verbatim}
%\usepackage{amsbsy}
\graphicspath{{C:/Users/janne.pitkaniemi/Kingstontikku/figures/}}

\setbeamertemplate{footline}[page number]
\DeclareGraphicsExtensions{.pdf, .jpg}
% \DeclareGraphicsExtensions{.eps}

\newcommand{\Rlogo}[1]{\includegraphics[#1]{Rlogo}}

 \input{usefulesa}

% \newcommand{\R}{\bf \textsf{R}}
\parskip\medskipamount



\title{Epidemiologic Data Analysis using R\\
Part 3: Basic analysis of rates}  % : \\ Analysis of Follow-up Studies}
% \Rlogo{height=1em} }

\author{Janne Pitk\"aniemi \\ (Esa L{\"a}{\"a}r{\"a})}

\institute{Finnish Cancer Registry, Finland,   
 \texttt{<janne.pitkaniemi@cancer.fi>} \\
 (University of Oulu, Finland,   
 \texttt{<esa.laara@oulu.fi>}) }

\date{University of Tampere \\Faculty of Social Sciences \\ % University of Tampere/Postgraduate training,
Feb 26- Apr 9  2018}


%\begin{center} \Rlogo{height=2em} \end{center}

\AtBeginSubsection[]
{
  \begin{frame}<beamer>
    \frametitle{Outline}
    \tableofcontents[currentsection,currentsubsection]
  \end{frame}
}

%\beamerdefaultoverlayspecification{<+->}

\begin{document}
\definecolor{darkgreen}{rgb}{0,.5,0}

\begin{frame}
    \titlepage
\end{frame}



%-------------------------------------------

\begin{frame}
\frametitle{Contents}
\ \\
\bi
\item[1.] Person-time data, hazard and incidence rates, 
\medskip
\item[2.] Comparative parameters of rates and their estimation, 
\medskip
\item[3.] Poisson regression models and comparative parameters, 
\medskip
\item[4.] Adjustment for confounding and evaluation of modification by Poisson regression,
\medskip
\item[5.] Goodness-of-fit evaluation.
\ei
Main R functions covered:
\bi
\item {\tt glm()}
\item tools for extracting results from a {\tt glm} model object
\ei
\end{frame}

%---------------------------------------

\begin{frame}[fragile]
 \frametitle{Person-time data and incidence rates}
\ \\
Summarized data on
outcome from cohort study, in which two exposure groups, as to binary risk factor $X$, have been followed-up over individually variable times.
\begin{center}
\begin{tabular}{ccc}
\toprule
Exposure to         &     Number of       &    Person-\\ 
 risk factor        &        cases        &        time\\
\midrule
      yes        &                        $D_{1}$ &        $Y_{1}$  \\
      no         &                        $D_{0}$  &        $Y_{0}$\\
\midrule    
     total        &                        $D_{\+}$  &  $Y_{\+}$\\
\bottomrule
\end{tabular}
\end{center}
Empirical \textbf{incidence rates} by exposure group:
$$ I_1 = {D_1}/{Y_1}, \qquad
    I_0 = {D_0}/{Y_0}. $$ 
These provide estimates for the true {\bf hazards} 
(or hazard rates) $\lambda_1$ and $\lambda_0$ assumed constant within exposure categories.
\end{frame} 

%---------------------------
\begin{frame}[fragile] \frametitle{Hazards and their comparison}
\ \\
Parameters of interest:
\bi
\item[(i)] \textbf{hazard ratio}
$$ \rho = \frac{\lambda_1}{\lambda_0} = 
\frac{ \mbox{hazard among exposed} } { \mbox{hazard among unexposed} }.
$$
\medskip
\item[(ii)] \textbf{hazard difference}
$$\delta = \lambda_1 - \lambda_0$$
\ei
Null hypothesis $H_0: \rho = 1 \Leftrightarrow \delta = 0$\\ $\Leftrightarrow$  exposure has no effect.
\end{frame} 

%-----------------------------------------

\begin{frame}[fragile] \frametitle{Estimation of hazard ratio}
\ \\
Point estimator of true hazard ratio $\rho$:
empirical\\ {\bf incidence rate ratio} (IR)
$$  \widehat\rho = \mbox{IR}  =  \frac{I_1}{I_0}
    = \frac{ D_{1} / Y_{1} }{ D_{0} / Y_{0} } 
    = \frac{ D_{1} / D_{0} }{ Y_{1} / Y_{0} }.         
$$
{\bf NB.} The last form is particularly useful:\\ 
 = {\bf exposure odds ratio} (EOR).

Standard error of log(IR),  95\% {error factor} \& 95\% CI for 
$\rho$:
\bes
 \SEL & = & \sqrt{ \frac{1}{D_1} + \frac{1}{D_0} } \\  
 \EF & = & \exp \{ 1.96 \times \SEL \} \\  
 \CI & = & [ \mbox{IR} / \EF ,\ \mbox{IR} \times \EF ] . 
\ees
{\bf NB.} Random error depends inversely on numbers of cases.
\end{frame} 

% --------------------------------------

\begin{frame}[fragile] \frametitle{Estimation of hazard difference}
\ \\
Point estimator of true hazard difference $\delta$:
empirical \\ {\bf incidence rate difference} (ID)
$$  \widehat\delta = \mbox{ID}  =  I_1 - I_0
    = \frac{ D_{1}}{ Y_{1} } - \frac{ D_{0}}{ Y_{0} } 
$$
Standard error of ID,  95\% {error margin} \& 95\% CI for 
$\delta$:
\bes
 \SE & = & \sqrt{ \frac{I_1^2}{D_1} + \frac{I_0^2}{D_0} } \\  \\
 \EM & = & 1.96 \times \SE \\  \\
 \CI & = & [ \mbox{ID} - \EM ,\ \mbox{ID} + \EM ] . 
\ees
NB. Random error again depends inversely on no. of cases.


\end{frame}

%----------------------------------------------------


\begin{frame}[fragile]
 \frametitle{Example. British doctors' study (Doll \& Hill '66)} 
 
CHD mortality in males
by smoking and age.\\  Cases ($D$),  person-years ($Y$), and mortality rates ($I$ per 
$10^4$ y).
\begin{center}
\begin{tabular}{@{\extracolsep{6pt}}lrrrrrr}
\toprule
\multicolumn{1}{c}{ } & 
\multicolumn{3}{c} {Smokers}         &        
\multicolumn{3}{c} {Non-smokers}  \\
\cmidrule(lr){2-4} \cmidrule(lr){5-7}
Age(y) & $D$ & $Y$ &    $I$ & $D$ & $Y$ &  $I$ \\
\midrule
35-44  & 32  & 52407  & 6   & 2   & 18790 & 1 \\
45-54  & 104 & 43248  & 24  & 12  & 10673 & 11 \\
55-64  & 206 & 28612  & 72  & 28  & 5710  & 49 \\
65-74  & 186 & 12663  & 147 & 28  & 2585  & 108 \\
75-84  & 102 & 5317   & 192 & 31  & 1462  & 212 \\
\midrule
Total  & 630 & 142247 & 44  & 101 & 39220 &  26  \\
\bottomrule
\end{tabular}
\end{center}
\end{frame} 

% --------------------------------------

\begin{frame}[fragile] \frametitle{Example (cont'd). }

Crude incidence rates: \\
$\qquad I_1 = 630/142247 \ \mbox{y} = 44.3$ per $10^4$ y, and\\
$\qquad I_0 = 101/39220 \ \mbox{y}\ = 25.8$ per $10^4$ y. 

Crude estimate of overall hazard ratio $\rho$ with SE, etc.
\bes
\widehat\rho      & = & \mbox{IR} = \frac{ 44.3 }{ 25.8 } = {\bf 1.72} \\ \\
\SEL & = & \sqrt{ \frac{1}{630} +  \frac{1}{101} } = {\bf 0.1072} \\ \\
              \EF & = & \exp(1.96\times 0.1072) =  {\bf 1.23}
\ees
95\% CI for $\rho$:
$ [ 1.72/1.23, \ 1.72\times 1.23 ]= [{\bf 1. 39},\ {\bf 2.12}] $ \\ 

Two-tailed $P < 0.001 $.
\end{frame} 


% ------------------------------------------

\begin{frame}[fragile] \frametitle{Poisson regression model for rate ratio}
\ \\
\bi
\item {\it Random part}: Number of cases in exposure group $j = 0,1$
$$ D_j \sim \mbox{Poisson}(\lambda_j Y_j) , $$
where $\mu_j = \lambda_j Y_j$ = {\it expected number} of cases.
\bigskip
\item {\it Systematic part \& link function}:\\ linear predictor $\alpha + \beta X_j$ with {\it logarithmic} (log) link 
$$ \log(\lambda_j) = \alpha + \beta X_j , $$
equivalently on the original hazard scale:
$$ \lambda_j = \exp(\alpha + \beta X_j) . $$
\ei
\end{frame} 


% ------------------------------------------

\begin{frame}[fragile] \frametitle{Poisson model for rate ratio (cont'd)}
\ \\
Interpretation,
\bi
\item[{ }] $X_j=$ $\begin{cases}
		1  & \text{if exposed } (j=1), \\
            0  & \text{if unexposed } (j=0),\\
		\end{cases}$
		\medskip
\item[{ }] $\alpha=$  $\log(\lambda_0)$, log-baseline rate,
\medskip
\item[{ }] $\beta=$ $\log(\rho) = \log(\lambda_1/\lambda_0)$, 
                logarithm of true hazard ratio,
     \medskip           
\item[{ }] $e^{\beta}=$ $\rho$ = true hazard ratio.
\ei
Special case of generalized linear models!
\end{frame} 

%----------------------------------------------------------------------

\begin{frame}[fragile] \frametitle{Example.  Crude analysis of CHD mortality in R}
\ \\
A ready data frame contains
\bi
\item four variables: 
  \bi
   \item[ ] {\tt age} = age group -- a {factor} with 5 levels,
   \medskip
   \item[ ]
        {\tt smok} = smoking: 1 = ``yes'', 0 = ``no'',\medskip
   \item[ ]        
        {\tt d} = number of cases, \medskip
   \item[ ]
        {\tt y} = person-years.
  \ei
\item 10 observations (one for each age-smoking combination).
\ei
\end{frame} 

%----------------------------------------------------------------------

\begin{frame}[fragile] \frametitle{Example.  Analysis of CHD rates (cont'd)}
\ \\
\begin{verbatim}
> data.frame(age, smok, d, y, rate)
     age smok   d     y  rate
 1 35-44    1  32 52407   6.1
 2 35-44    0   2 18790   1.1
 3 45-54    1 104 43248  24.0
 4 45-54    0  12 10673  11.2
 5 55-64    1 206 28612  72.0
 6 55-64    0  28  5710  49.0
 7 65-74    1 186 12663 146.9
 8 65-74    0  28  2585 108.3
 9 75-84    1 102  5317 191.8
10 75-84    0  31  1462 212.0
 
\end{verbatim}
\end{frame} 

%-----------------------------
\begin{frame}[fragile] \frametitle{Fitting Poisson model for crude rate ratio}
\ \\
Poisson model with log-link (default) for crude rates
\begin{verbatim}
> m1 <- glm(d/y ~ smok, family=poisson(), weights=y)
\end{verbatim}
Output (shortened) using {\tt summary()} method
\small
\begin{verbatim}
> summary(m1)
Coefficients:
            Estimate Std. Error z value Pr(>|z|)
(Intercept) -5.96182    0.09944 -59.953  < 2e-16
smok         0.54222    0.10713   5.062 4.16e-07

Null deviance:     935.07 on 9 degrees of freedom
Residual deviance: 905.98 on 8 degrees of freedom
\end{verbatim}
\normalsize
\end{frame} 

%-----------------------------
\begin{frame}[fragile] 
\frametitle{Fitting crude rate ratio model (cont'd)}
\ \\
Main results: 
\bi
\item[$\widehat\alpha$] = $ -5.96 = \log(25.8/10^4 \mbox{ y}), \quad$ (SE = 0.10),
\item[$\widehat\beta$] = $0.54 = \log(1.72), \quad$ (SE = 0.11)
\ei

Function {\tt ci.lin()} transforms results to ratio scale
\small
\begin{verbatim}
> round(ci.lin(m1, Exp=T)[, -(3:4)], 4 )
           Estimate StdErr exp(Est.)   2.5%  97.5%
(Intercept) -5.9618 0.0995    0.0026 0.0021 0.0031
smok         0.5422 0.1072    1.7198 1.3940 2.1219
\end{verbatim}
\normalsize
Compare the results with those obtained above using simple
estimation \& SE formulas.
\end{frame} 

%-------------------------------
\begin{frame}[fragile] \frametitle{Fitting crude rate ratio model (cont'd)}
\ \\
The Poisson model above can also be fitted as follows:
  
\verb|> glm(d ~ smok, fam =poisson(), offset=log(y))|

\bigskip
Here {\tt offset} refers to the logarithm of person-years {\tt y} in formula for expected 
numbers of cases $\mu_j = \lambda_j \times Y_j$:
\bes \log(\mu_j) & = & \log(\lambda_j Y_j)
                   =   \log(Y_j) + \log(\lambda_j) \\ 
    & = & \log(Y_j) + \alpha + \beta X_j , 
\ees

$\log(Y_j)$ is an {\bf offset} term in the linear predictor, meaning that
it has a fixed value 1 for the regression coefficient. 
\end{frame} 


%------------------------------------------

\begin{frame}[fragile] \frametitle{Stratified analysis}
\ \\%[-30pt]

Stratification of cohort data with person-time\\ -- 
at each level $k$ of covariate $Z$ results are summarized:
\begin{center}
\begin{tabular}{ccc}
\toprule
Exposure to & Number of         & Person- \\
risk factor & cases       & time\\
\midrule
 yes        & $D_{1k}$    & $Y_{1k}$ \\
 no         & $D_{0k}$    & $Y_{0k}$ \\
\midrule    
Total       & $D_{\+ k}$  & $Y_{\+ k}$ \\
\bottomrule
\end{tabular}
\end{center}
Stratum-specific rates by exposure group: 
$$ I_{1k}  = \frac{ D_{1k} }{ Y_{1k} } , \qquad
   I_{0k}  = \frac{ D_{0k} }{ Y_{0k} } .$$
\end{frame} 

%------------------------------------------

\begin{frame}[fragile] \frametitle{Stratum-specific comparisons}
\ \\
Let $\lambda_{jk}$ be true rate for exposure group $j$ ($j=0,1)$\\ 
and stratum $k$ ($k=0, \dots, K$). 
Let also
$$ \rho_k = \frac{ \lambda_{1k} } { \lambda_{0k} }, \qquad \delta_k = \lambda_{1k} - \lambda_{0k}$$

be the rate ratios and rate differences between the\\ exposure groups 
in stratum $k$.

Two simple models assuming homogeneity:
\bi
\item common rate ratio: $\rho_k = \rho$ for all $k$,
\item common rate difference: $\delta_k = \delta$ for all $k$.
\ei
Only one of these can in principle hold.  However, almost always neither homogeneity assumption is exactly true.
\end{frame} 

 
%-------------------------------
\begin{frame}[fragile]
 \frametitle{Example. British male doctors (cont'd)} 

\ \\
CHD mortality rates (per $10^4$ y) and numbers of cases ($D$)
by age and cigarette smoking.

\medskip
Mortality rate differences (ID) and ratios (IR) in age strata.
\begin{center}
\begin{tabular}{@{\extracolsep{6pt}}lrrrrrr}
\toprule
& \multicolumn{2}{c} {Smokers} &        
\multicolumn{2}{c} {Non-smokers} \\
\cmidrule(lr){2-3} \cmidrule(lr){4-5}
Age (y) & rate & ($D$) & rate & ($D$) & ID &  IR         \\
\midrule
35-44  & 6.1  & (32)  & 1.1  & (2)   & 5  &  5.7 \\
45-54  & 24   & (104) & 11   & (12)  & 13 &  2.1 \\
55-64  & 72   & (206) & 49   & (28)  & 23 &  1.5 \\
65-74  & 147  & (186) & 108  & (28)  & 39 &  1.4 \\
75-84  & 192  & (102) & 212  & (31)  & $-$20 &  0.9 \\
\midrule
Crude  &  44  & (630) &  26  & (101) & 18 & 1.7 \\ 
\bottomrule
\end{tabular}
\end{center}
\end{frame} 

%-------------------------------
\begin{frame}[fragile] \frametitle{Example (cont'd).}
\ \\ 
\bi
\item Both types of comparative parameter, rate ratios $\rho_k$ and rate differences $\delta_k$ appear heterogenous, because
\bi
\item ID increases by age -- at least up to 75 y,
\item IR decreases by age.
\ei
\item Part of this observed heterogeneity may be due to random variation. 
\item Yet, any single-parameter comparison by common rate ratio or rate difference  
may not adequately capture the
joint pattern of true rates.
\ei

$\quad \Rightarrow$ Effect modification must be evaluated.

\end{frame} 

%---------------------------------------------------------------------


\begin{frame}[fragile] \frametitle{Rate ratio adjustment by Poisson model}
\ \\
Define Poisson regression model for
\bi
\item one binary exposure variable $X$ and 
\item
one categorical (polytomous)  factor $Z$.
\ei
{\it Random part}: No. of cases in exposure group $j$ $(j=0,1)$
and covariate level $k$ $(k = 1, \dots, K)$ is
$ D_{jk} \sim \mbox{Poisson}(\lambda_{jk} Y_{jk})$ 

{\it Systematic part}: 
$ \log(\lambda_{jk}) = \alpha + \beta X_j +
  \gamma_k  , $
where $X_j$ is (0/1),

\bi
\item[{ }] $\alpha=$ $\log(\lambda_{01})$ = log-baseline rate, 
\item[{ }] $\gamma_k=$ $\log(\lambda_{jk}/\lambda_{j1})$,
\item[{ }] $\beta=$ $\log(\rho) = \log(\lambda_{1k}/\lambda_{0k})$,
\item[{ }] $e^{\beta}=$ $\rho$ = true rate ratio for the effect of exposure to $X$.
\ei

{\it How do we read this?}
\end{frame} 

%----------------------------------------------

\begin{frame}[fragile] 
\frametitle{Implications of model definition}

\bi
\item homogeneity of true rate ratio $\rho_k = \rho$ for $X$ \\
 across levels of $Z$ is 	assumed,
	\medskip 
\item inclusion of $Z$ leads to adjustment for $Z$ in estimating the common effect of $X$,
\medskip 
\item $e^{\gamma_k}$ = rate ratio for level $k$ of $Z$ {\it vs.} level 1
      is the same in both exposure groups ($j=0,1$) \\
      $\Rightarrow$ homogeneity of
      the effect of $Z$ is assumed, too.
 \medskip     
\item level $k=1$ is chosen as the {\it reference} level for $Z$\\ (like 
	``unexposed'' is reference for $X$), 
\medskip	
\item before model fitting, binary {\it indicator variables} $Z_k$ 
	for levels $k=1, \dots, K$ of $Z$ must be defined:
	$$ Z_k  =\begin{cases}
		   1, & \mbox{if observation belongs to level } k, \\
               0, & \mbox{ otherwise}.
		   \end{cases}
	$$
\ei 
\end{frame} 


%-----------------------------

\begin{frame}[fragile] \frametitle{Example.  CHD in British doctors (cont'd)} 

Factor {\tt age} has $5$ levels. \\
Indicator variables for each age level are generated in R when defining the model, and the following model matrix is returned.
\small
\begin{verbatim}
> cbind(data.frame(age), model.matrix(m2))
     age (I't) age45-54 age55-64 age65-74 age75-84 smok
 1 35-44    1         0        0        0        0    1
 2 35-44    1         0        0        0        0    0
 3 45-54    1         1        0        0        0    1
 4 45-54    1         1        0        0        0    0
 5 55-64    1         0        1        0        0    1 
 6 55-64    1         0        1        0        0    0
 7 65-74    1         0        0        1        0    1
 8 65-74    1         0        0        1        0    0
 9 75-84    1         0        0        0        1    1
10 75-84    1         0        0        0        1    0
\end{verbatim}
\normalsize
\end{frame} 


%-----------------------------

\begin{frame}[fragile] \frametitle{Fitting the adjustment model} 
\ \\
Define a new model object {\tt m2} with {\tt age} and {\tt smok}
\small
\begin{verbatim}
> m2 <- glm( d/y ~ age + smok,
+            family=poisson(link=log), weights=y)}
> summary(m2)

Coefficients:            
                Estimate Std. Error z value Pr(>|z|)
(Intercept)      -7.9193     0.1917 -41.301  < 2e-16
age 45-54         1.4840     0.1951   7.607 2.81e-14
age 55-64         2.6275     0.1837  14.302  < 2e-16
age 65-74         3.3505     0.1848  18.132  < 2e-16
age 75-84         3.7001     0.1922  19.251  < 2e-16
smok              0.3545     0.1074   3.302  0.00096

Residual deviance: 12.132 on 4 degrees of freedom
\end{verbatim}\normalsize
\end{frame} 

%--------------------------

\begin{frame}[fragile] \frametitle{Fitting adjustment model (cont'd)}
\ \\
Results on the ratio scale
\begin{verbatim}
> round(ci.lin(m2, Exp=T)[, 5:7], 4 )
               exp(Est.)    2.5%   97.5%
(Intercept)       0.0004  0.0002  0.0005
age 45-54         4.4106  3.0091  6.4648
age 55-64        13.8392  9.6546 19.8375
age 65-74        28.5168 19.8524 40.9627
age 75-84        40.4512 27.7541 58.9571
smok              1.4255  1.1550  1.7594
\end{verbatim}

$\Rightarrow$ Age-adjusted rate ratio [95\% CI] for smoking:
$$\widehat \rho = {\bf 1.43} \ [{\bf 1.16}, {\bf 1.76}]$$

\end{frame} 

%-------------------------------------------------------------------

\begin{frame}[fragile] \frametitle{Fitted values \& residuals}

\ \\
From the estimated coefficients we can calculate \\
 {\bf fitted}
 linear predictors $\widehat \eta_{jk}$, {hazards} $\widehat \lambda_{jk}$ 
and {numbers} $\widehat \mu_{jk}$:
\bes 
  \widehat \eta_{jk} & = & \widehat\alpha + \widehat\beta x + \widehat\gamma_k \\
   \widehat \lambda_{jk} & = & \mbox{exp}(\widehat \eta_{jk}), \quad \widehat\mu_{jk}     =  \widehat\lambda_{jk} Y_{jk} 
\ees
In R the two first can be extracted directly from the\\ fitted model object {\tt m2}:
\small
\begin{verbatim}
   > fit.eta <- m2$linear.predictor
   > fit.rate <- fitted(m2); fit.mu <- y*fit.rate
\end{verbatim}
\normalsize
{\bf NB}.  If count {\tt d} is declared as response with {\tt log(y)} as offset, then {\tt fitted()}
returns the fitted numbers of cases $\widehat \mu_{jk}$.
\end{frame} 

%-----------------------
\begin{frame}[fragile] \frametitle{Fitted values \& residuals (cont'd)}

\ \\ 
{\bf Deviance residual} for cell $jk$ ({\tt resid(m2)} in R):
$$ d_{jk}  =  \mbox{sign}(Y_{jk} - \widehat\mu_{jk}) 
                    \times 
                   \sqrt{ 2 \left\lbrace
                         Y_{jk} \log \left( \frac{ Y_{jk} }{ \widehat\mu_{jk} } \right)
                         - (Y_{jk} - \widehat\mu_{jk}) \right\rbrace } 
$$
{\bf Pearson residual} ({\tt resid(m2, type="pearson")}):
$$ r_{jk} = \frac{ Y_{jk} -  \widehat\mu_{jk} }
          {\sqrt{  \widehat\mu_{jk} } } . $$
Small value of either residual\\  
$\rightarrow$ consistency of observation with model.

``Large'' (in absolute value) residual\\
$\rightarrow$ lack of fit for that cell.
\end{frame} 

%----------------
\begin{frame}[fragile] \frametitle{Example.  Fitted values \& residuals}
 
\small
\begin{verbatim}
> data.frame(age,smok,rate,fit.rate,d,fit.mu,res.D,res.P)
     age smok  rate fit.rate   d fit.mu res.D res.P
1  35-44    1   6.1      5.2  32   27.3  0.90  0.93
2  35-44    0   1.1      3.6   2    6.8 -2.18 -1.85
3  45-54    1  24.0     22.9 104   99.0  0.51  0.51
4  45-54    0  11.2     16.0  12   17.1 -1.31 -1.24
5  55-64    1  72.0     71.7 206  205.1  0.05  0.05
6  55-64    0  49.0     50.3  28   28.7 -0.14 -0.14
7  65-74    1 146.9    147.8 186  187.2 -0.09 -0.09
8  65-74    0 108.3    103.7  28   26.8  0.23  0.23
9  75-84    1 191.8    209.7 102  111.5 -0.91 -0.90
10 75-84    0 212.0    147.1  31   21.5  1.92  2.05
\end{verbatim}
\normalsize

NB! Fitted rate ratios between smokers and non-smokers:
$$ \frac{5.2}{3.6} = \dots = \frac{209.7}{147.1} = 1.43 \ \text{at each age level}$$
\end{frame} 

%---------------------------
\begin{frame}[fragile]
 \frametitle{Summary goodness-of-fit statistics}
\ \\ 
\bi
   \item {\bf Deviance} (Dev)  
         = sum of squares of deviance residuals,
   \item {\bf Pearson's} $X^2$ (chi-square) 
          = s.s. of Pearson residuals  
\ei
Assume that 
\bi
\item the model is ``true'', and 
\item the data are not {\it sparse}, i.e. numbers of cases in each line
      of the (grouped) data matrix are sufficiently big.
\ei
Under these assumptions 
\bi
\item  residuals $d_{jk}$ and $r_{jk}$ all approximately Normal(0,1),
\item  Both Dev and $X^2$ are approximately 
       $\chi^2$-distributed with 
       {\it degrees of freedom} (df) being equal to the \\
       {\it residual} df = no. of obs -- no. of coeff's 
\item Expected value of these statistics 
      = residual df.
\ei
\end{frame} 
%-------------------------
\begin{frame}[fragile] \frametitle{Goodness of fit (cont'd)}
\ \\
In our example 
\bi
\item ``large'' residuals in extreme age groups,
\item $X^2$ and Dev 2.8 to 3 times higher than df.
\ei
$\rightarrow$ Some lack of fit with assumed model.

($P$-values: P($X^2$ $>$ 11.2) = 0.025, and \\
  P(Dev $>$ 12.1) = 0.017)
\end{frame} 
%-------------------------
\begin{frame}[fragile]
 \frametitle{Possible causes for lack of fit}
\bi
\item Unrealistic random part (Poisson)?
\medskip
\item Misspecified systematic part?
      \bi
      \item[--] Wrong functional form for effects? \medskip
      \item[--] Missing important risk factors? \medskip
      \item[--] Failure to take into account {effect modification}
       or ``interaction''.
      \ei
\ei
In this example the last possibility is very likely:  \\ 
the rate ratios are clearly heterogenous. 

\medskip
Yet, inclusion of 4 interaction parameters
would lead to a saturated model.
\end{frame} 


%-----------------------------

\begin{frame}[fragile] \frametitle{Continuous covariates}
\ \\ 
Stratified analysis is adequate with
\bi
\item qualitative exposure factor with few levels,
\item few qualitative covariates with few levels.
\ei
Effects of continous factors can often be 
described by smooth functions with small number of parameters.

Stratification requires categorization of 
continuous variables and level specific parameters 

$\Rightarrow$ loss of efficiency.
\end{frame} 

%-----------------------------

\begin{frame}[fragile] \frametitle{Continuous covariates (cont'd)}
\ \\
Regression models can accommodate functions of 
continuous variables 
\bi
\item[$+$] more parsimonious models, \medskip 
\item[$+$] greater flexibility in modelling, \medskip
\item[$-$] more opportunities for misspecification,
{\it i.e.} creation of ``wrong'' 
models.
\ei
\end{frame} 

%--------------------------------------------

\begin{frame}[fragile] \frametitle{Example (cont'd)}
\ \\
Treat age as continuous variable taking midpoint of each ageband
(40, 50, \dots, 80)
as quantitative age value. 

Include linear and quadratic term of age applying 
\bi
\item centering at 60 years, and 
\item scaling, \textit{i.e.} division by 10 years
\ei
This is achieved by
\small
\begin{verbatim}
   > A.L <- as.numeric(age)-3 
   > A.Q <- A.L*A.L - 2
\end{verbatim}
\normalsize   
\end{frame} 


%--------------------------------------------

\begin{frame}[fragile] \frametitle{Example (cont'd)}
\ \\ 
Model with linear and quadratic terms for age, and
interaction between linear term of age and smoking:
\small
\begin{verbatim}
> m3 <- glm( d/y ~ A.L + A.Q + smok + A.L:smok,
+           family=poisson( ), weights=y )
> round(ci.lin(m3, Exp=T)[, -(3:4)], 4 )

           Estimate StdErr exp(Est.)   2.5%  97.5%
(Intercept) -5.8368 0.1213    0.0029 0.0023 0.0037
A.L          1.1904 0.0923    3.2885 2.7441 3.9408
A.Q         -0.1977 0.0274    0.8206 0.7778 0.8659
smok         0.5183 0.1262    1.6792 1.3112 2.1505
A.L:smok    -0.3075 0.0970    0.7352 0.6079 0.8893
\end{verbatim}
\normalsize
Interpretation of parameters?

\end{frame} 

%-----------------------
\begin{frame}[fragile] \frametitle{Example (cont'd)}
\ \\
Observed and fitted rates \& fitted 
rate ratios between smokers and non-smokers
in each ageband
{\small
\begin{verbatim}
     age smok  rate fit.rate   d fit.mu res.D res.P
1  35-44    1   6.1      5.6  32   29.3  0.44  0.44
2  35-44    0   1.1      1.8   2    3.4 -0.83 -0.77
3  45-54    1  24.0     24.7 104  106.8 -0.27 -0.27
4  45-54    0  11.2     10.8  12   11.5  0.13  0.13
5  55-64    1  72.0     72.8 206  208.3 -0.15 -0.15
6  55-64    0  49.0     43.3  28   24.7  0.64  0.65
7  65-74    1 146.9    144.4 186  182.9  0.23  0.23
8  65-74    0 108.3    116.9  28   30.2 -0.41 -0.41
9  75-84    1 191.8    192.9 102  102.6 -0.06 -0.06
10 75-84    0 212.0    212.5  31   31.1 -0.01 -0.01
\end{verbatim}
\normalsize
Residual Dev = 1.64  df = 5; The fit is excellent! 
}
\end{frame} 

\begin{frame}

\frametitle{Some closing remarks}
\bi
\item
  Hazard differences can also be estimated by
Poisson-modelling, because {\tt link='identity'}
can be coupled with {\tt family=poisson}  in {\tt glm()}.
\medskip
\item
  Poisson models can also be fitted on ungrouped data.
Units of observations may even be shorter intervals of
total individual follow-up times: 
\bi
\item
Each member of the
study population can contribute several separate
time-slots as observational units. 
\item
More about time-splitting in the next lectures.
\ei
\item
  With continuous covariates it may be difficult to keep
hazards positive when the \textit{additive hazards}
model with \texttt{link='identity'} is attempted to be fitted.
\ei
\end{frame}

\end{document}
%-----------------------------

