% typeset using LaTeX2e
%\documentclass[handout, 12pt]{beamer}
\documentclass[handout,12pt]{beamer}
%\mode<presentation>
%{
% \usetheme{Singapore}
% \setbeamercovered{transparent}
%}
% making handouts
%\usepackage[overlay]{textpos}
%\usepackage{pgfpages}
%\pgfpagesuselayout{4 on 1}[letterpaper, border shrink = 5mm,landscape]
%\mode<presentation>{
%  \setbeamercovered{invisible}
%}
% end of making handouts

\usepackage[latin1]{inputenc}
\usepackage[english]{babel}
\usepackage{epsfig}
%\usepackage{rotating}
\usepackage{graphicx}
%\usepackage{mslapa}
\usepackage{amsmath}
%\usepackage[all]{xy}
% \usepackage{amssymb,graphicx}
%\input psfig.sty
\usepackage{booktabs}
\usepackage{graphpap}
\usepackage{verbatim}
%\usepackage{amsbsy}
%\graphicspath{{C:/Users/janne.pitkaniemi/Kingstontikku/figures/}}
\graphicspath{{C:/Users/janne.pitkaniemi/figures/}}

\setbeamertemplate{footline}[page number]
\DeclareGraphicsExtensions{.pdf, .jpg, .png,.jpeg}
% \DeclareGraphicsExtensions{.eps}

\newcommand{\Rlogo}[1]{\includegraphics[#1]{Rlogo}}

\input{usefulesa}
\usepackage[labelformat=empty]{caption}
\captionsetup[figure]{labelformat=empty}

% \newcommand{\R}{\bf \textsf{R}}
\parskip\medskipamount

\title{Epidemiologic Data Analysis using R \\
Part 4: Time-splitting in cohort studies}  % : \\ Analysis of Follow-up Studies}
% \Rlogo{height=1em} }

\author{Janne Pitk\"aniemi \\ (Esa L{\"a}{\"a}r{\"a})}

\institute{Finnish Cancer Registry, Finland,   
 \texttt{<janne.pitkaniemi@cancer.fi>} \\
 (University of Oulu, Finland,   
 \texttt{<esa.laara@oulu.fi>}) }

\date{University of Tampere \\Faculty of Social Sciences \\ % University of Tampere/Postgraduate training,
Feb 26- Apr 9  2018}


%\begin{center} \Rlogo{height=2em} \end{center}

\AtBeginSubsection[]
{
  \begin{frame}<beamer>
    \frametitle{Outline}
    \tableofcontents[currentsection,currentsubsection]
  \end{frame}
}

%\beamerdefaultoverlayspecification{<+->}

\begin{document}
\definecolor{darkgreen}{rgb}{0,.5,0}

\begin{frame}
    \titlepage
\end{frame}


%-------------------------------------------

\begin{frame}
\frametitle{Contents}
 
\bi
\item[1.] Basic concepts of time to event analysis
\medskip
\item[2.] Piecewise constant hazards model and \\
age-specific incidence rates
\medskip
\item[3.] Splitting follow-up times by age
\medskip
\item[4.] Accounting for current age in rate ratio estimation 
\ei

Main R functions to be covered, all in \texttt{Epi} package
\bi
\item {\tt Lexis()} 
\item {\tt splitLexis()}
\item {\tt timeBand()} 
\ei
\end{frame}
%---------------------------------------

\begin{frame} \frametitle{Time to event analysis}
\ \\
Analysis of incidences = analysis of {\it times to event} or {\it failure times} or {\it survival times} (censored).
 
Mathematical concepts:
\bes
    T & = & \mbox{ time to outcome event --  random variable,} \\
 S(t) & = & P(T>t) = \mbox{ {\bf survival} function of $T$,} \\
     & = & \mbox{ probability of avoiding 
the event up to given time }t,\\
\lambda(t) & = & -S'(t)/S(t) = 
\mbox{ {\bf intensity} or {\bf hazard} function}, \\
\Lambda(t) & = & \int_0^t \lambda(u)du = - \log S(t)
   = \mbox{\bf cumulative hazard}, \\
  F(t) & = & 1-S(t) = 1-\exp\{ -\Lambda(t) \} 
  = \mbox{{\bf risk} function} \\
    & = & \mbox{ probability of the outcome to occur by }t   \\
    & = & \mbox{ cumulative distribution function of }T.
\ees

\end{frame} 

%-------------------------------------

\begin{frame} \frametitle{Hazard rate or intensity function}
\ \\
Can be viewed as \textit{theoretical incidence rate}. Formally
$$ \lambda(t) = \lim_{\Delta \rightarrow 0}
 \frac{P(t < T \leq t+\Delta \mid T > t)}{\Delta} $$
\bi
\item[$\approx$] Probability of failure occurring 
in a short interval $]t, t+ \Delta]$, given ``survival'' or
avoidance of event \\ 
up to its start $t$,  divided by the interval length.
\ei
This is equivalent to saying that over this short interval
$$ \mbox{risk } \approx  \mbox{ rate }\times 
\mbox{ length of interval} $$
or $\qquad P(t < T \leq t+\Delta \mid T > t) \ 
 \approx \ \lambda(t)\times \Delta$.
\end{frame} 

%----------------------------

\begin{frame} \frametitle{Exponential or constant hazard model}
\ \\
Simplest probability model for time to event: \\
 {\bf Exponential distribution}, Exp($\lambda$), in which
$$
 \lambda(t) = \lambda \mbox{  (constant)}\quad 
\Rightarrow\quad 
 \Lambda(t) = - \log S(t) = \lambda t
$$
Analysis of failure data of $n$ individuals. 
For subject $i$ let
\bes
y_i & = & \mbox{ time to event or 
time to censoring,}\quad
  Y = \sum y_i \\
d_i & = & \mbox{ indicator for 
observing the event,}\quad
  D = \sum d_i
\ees
Exp($\lambda$) model $\Rightarrow$ 
\textbf{Likelihood function} of $\lambda$ is
$$ L(\lambda) = \prod_{i=1}^n \lambda(y_i)^{d_i} S(y_i)
     =  \prod_{i=1}^n \lambda^{d_i} e^{-\lambda y_i}
             \ = \ \exp( D \log \lambda -\lambda Y) $$
\end{frame} 

%--------------------------------------

\begin{frame} \frametitle{Constant rate -- Poisson model}
\ \\ 
This is actually equivalent to the \textit{Poisson-likelihood},
\textit{i.e.} likelihood of $\lambda$
assuming that the number of cases 
$D$ is distributed according to the \textbf{Poisson
distribution} with expected value $\lambda Y$.

\bigskip
With randomly censored exponential times 
$D$ is only approximately Poisson. This is sufficient, though,
 for likelihood-based (\& asymptotic frequentist) inference.
 
 \bigskip
Solving the {\it score equation}: \ $d \log L(\lambda)/d\lambda = 0$ \\
$\rightarrow$ 
{\bf maximum likelihood estimator} (MLE) of $\lambda$ is
$$
 \widehat\lambda = \frac{D}{Y}  = 
  \frac{\mbox{number of cases}}
  {\mbox{total person-time}} 
  = \mbox{ empirical incidence rate!}
$$
\end{frame} 

%--------------------

\begin{frame} 
\frametitle{Time to event -- when to start the clock?}
\ \\
Incidence can be studied on various time scales, {\it e.g.}
\bi
\item age (starting point = birth),
\item exposure time (first exposure),
\item follow-up time (entry to study),
\item duration of disease (diagnosis).
\ei 
Age is usully the strongest time-dependent 
determinant of health 
outcomes.

\medskip
 Age is also often correlated with
duration of ``chronic'' exposure ({\it e.g.} years of smoking).

\medskip
Therefore, adjustment for {\it current age} is needed rather than
for {\it age at entry} to follow-up (like in clinical survival studies).
\end{frame} 

%-------------------------------

\begin{frame} \frametitle{Age to event split into agebands}
\ \\

Let $T$ = age at which outcome event occurs. \\
Parametric form of $\lambda(t)$, hazard by age -- usually unknown.

\medskip
{\bf Piecewise exponential model} or \textbf{piecewise constant hazards' model} -- an approximation for $\lambda(t)$:
$$ \lambda(t) = \lambda_k, \qquad t \in \ ]a_{k-1}, a_k],\quad \Delta_k = a_k - a_{k-1}, $$
where cutpoints $0 = a_0 < a_1 < \dots < a_K$ divide the age range into
disjoint {\bf agebands}, each with constant rate.

\medskip
In chronic disease epidemiology agebands with 
$\Delta_k = 5$ years (0-4, 5-9, $\dots$, 80-84)
or 10 years are commonly used.
\end{frame} 

%--------------------------------------------

\begin{frame} \frametitle{Age-specific incidence rates}
\ \\
For empirical estimation of rates we calculate 
in each ageband
\bes
 D_k & = & \mbox{ number of cases occurring in ageband } k, \\
 Y_k & = & \sum_{i=1}^n y_{ik} = \mbox{ total person-time in ageband } k,
\ees
where $y_{ik}$ is the time slot that subject $i$ spends
in ageband $k$ out of his/her whole {\bf follow-up
time} (from {\bf entry} to {\bf exit}).

ML estimators of  $\lambda_1, \dots, \lambda_K$: 
{\bf age-specific incidence rates}
$$ \widehat\lambda_k = I_k = {D_k}/{Y_k}, \quad k = 1, \dots, K $$
based on log-likelihood 
$\ \ \log L = \sum_{k} (D_k \log \lambda_k - \lambda_k Y_k)$. 
\end{frame} 

%----------------------------------------

\begin{frame} \frametitle{Cumulative rates \& risks}

In this model, the cumulative hazard and risk functions are
\bes
 \Lambda (t) & = & \sum_{a_j < t} \lambda_j \Delta_j + \lambda_k(t - a_{k-1}),\qquad t \in \ ]a_{k-1}, a_k] \\
 F(t) & = & 1 - S(t) = 1 - \exp\{ - \Lambda(t) \}, 
\ees
the latter assuming that no {\bf competing risks} are present.

\medskip
Estimation: Plug in empirical rates $\widehat\lambda_j = D_j/Y_j$ to get
the cumulative rate $C$ and incidence proportion $R$ by $t$:
\bes
  {C} & = & \widehat{\Lambda} (t) =  \sum_{a_j < t} \widehat\lambda_j \Delta_j 
  + \widehat\lambda_k (t - a_{k-1}),\qquad t \in \ ]a_{k-1}, a_k] \\
  R & = & \widehat{F}(t)  = 1 - \widehat{S}(t) = 1 - \exp\{ - \widehat{\Lambda}(t) \}
\ees

\end{frame}  

%---------------------------------------

\begin{frame} \frametitle{Example: Follow-up of a small geriatric cohort} 
\setlength{\unitlength}{1.2pt}
\begin{center}
\includegraphics[width=10cm,height=6cm ]{lifelines}
\end{center}
No's of cases/p-years \& rates (/100 y) in 5-y agebands:
$$ {1}/{21} = 4.8, \quad {1}/{16} life= 6.2, \quad {2}/{16.5} = 12.1$$
% \bes
% D_1 & = & 1, \quad D_2 =  1, \quad D_3  =  2 \\
% Y_1 & = & 14, \ \ Y_2  = 17, \ \  Y_3  =  14 \mbox{ years}\\
% I_1 & = & 7.1, \ \ I_2  = 5.9, \ \ I_3  =  14.3 \mbox{ per 100 y} 
% \ees
\end{frame} 

 \setlength{\unitlength}{1.4pt}

%--------------------------------------------

\begin{frame}[fragile]
 \frametitle{Splitting follow-up by {\tt Lexis()} in package {\tt Epi} }

Individual ages at entry and at exit, as well as outcomes
are assigned into vectors and stored in a data frame \texttt{coh}:  
\small
\begin{verbatim}
> ag.entry <- c(69, 70, 70.5, 71, 72, 76.9, 81, 81.9)
> ag.exit <- c(83.5, 74.5, 76, 77, 85, 82, 84, 85)
> event <- c(0,1,0,1,0,1,1,0) ; ind <- 1:8 
> coh <- data.frame( ind, ag.entry, ag.exit, event)
\end{verbatim}
\normalsize
Function {\tt Lexis()} specifies the
time scale(s) to be considered. It 
 creates an enriched data frame
belonging to class \texttt{Lexis}.
\small
\begin{verbatim}
> coh.L <- Lexis(entry = list(age = ag.entry), 
+                 exit = list(age = ag.exit), 
+          exit.status = event, data = coh, id = ind)
\end{verbatim}
\normalsize

\end{frame}

\begin{frame}[fragile]
\frametitle{Data frame of class \texttt{Lexis} }

\scriptsize
\begin{verbatim}
> coh.L
   age lex.dur lex.Cst lex.Xst lex.id ind ag.entry ag.exit event
1 69.0    14.5       0       0      1   1     69.0    83.5     0
2 70.0     4.5       0       1      2   2     70.0    74.5     1
3 70.5     5.5       0       0      3   3     70.5    76.0     0
4 71.0     6.0       0       1      4   4     71.0    77.0     1
5 72.0    13.0       0       0      5   5     72.0    85.0     0
6 76.9     5.1       0       1      6   6     76.9    82.0     1
7 81.0     3.0       0       1      7   7     81.0    84.0     1
8 81.9     3.1       0       0      8   8     81.9    85.0     0
\end{verbatim}
\normalsize
Interpretation of new columns
\begin{align*}
 \texttt{age} & =  \text{age at entry to follow-up},\\
 \texttt{lex.dur} & =  \text{duration of follow-up},\\
 \texttt{lex.Cst} & =  \text{current status at entry}, \\
  \texttt{lex.Xst} & =  \text{status at exit from follow-up}.
\end{align*}

\end{frame}

\begin{frame}[fragile]
\frametitle{Splitting follow-up times by agebands}

\medskip
Function \texttt{splitLexis()} splits individual follow-up times
into given agebands and expands the data frame.
\small
\begin{verbatim}
> coh.A <- splitLexis(coh.L, 
+       br = c(70,75,80,85), time.scale="age")
\end{verbatim}
\scriptsize
\begin{verbatim}
> coh.A
   lex.id  age lex.dur lex.Cst lex.Xst ind ag.entry ag.exit event
1       1 69.0     1.0       0       0   1     69.0    83.5     0
2       1 70.0     5.0       0       0   1     69.0    83.5     0
3       1 75.0     5.0       0       0   1     69.0    83.5     0
4       1 80.0     3.5       0       0   1     69.0    83.5     0
5       2 70.0     4.5       0       1   2     70.0    74.5     1
6       3 70.5     4.5       0       0   3     70.5    76.0     0
7       3 75.0     1.0       0       0   3     70.5    76.0     0
...
13      6 76.9     3.1       0       0   6     76.9    82.0     1
14      6 80.0     2.0       0       1   6     76.9    82.0     1
15      7 81.0     3.0       0       1   7     81.0    84.0     1
16      8 81.9     3.1       0       0   8     81.9    85.0     0
\end{verbatim}

\normalsize

\end{frame}


\begin{frame}[fragile]

\frametitle{Splitted \texttt{Lexis} object}

\bi
\item
Function \texttt{splitLexis()} expanded the original data frame such that
for all cohort members one or more rows were created, one for
each ageband into which a subject contributes person time.
\medskip
\item
Ex: Subject 1 has been under follow-up in all agebands considered, but subjects 7 and 8 only in 80$-<85$ y.
\medskip
\item
Function \texttt{timeBand()} 
converts variable \texttt{age} into 
factor \texttt{ageband}. Also, shorthand
names for person-time slots and occurrence of outcome event are given.
\small
\begin{verbatim}
> coh.A$ageband <- timeBand(coh.A, "age", "factor") 
> coh.A$y_ik <- coh.A$lex.dur # person-time slot
> coh.A$d_ik <- coh.A$lex.Xst # occurrence of outcome
\end{verbatim}
\normalsize
\ei

\end{frame} 



\begin{frame}[fragile]
\frametitle{Splitted \texttt{Lexis} object (cont'd)}
  
\begin{columns}[t] 
    \begin{column}{0.40\textwidth}
\scriptsize
\begin{verbatim}
 > coh.A[, c(1,10:12)] 
   lex.id   ageband y_ik d_ik
1       1 (-Inf,70]  1.0    0
2       1   (70,75]  5.0    0
3       1   (75,80]  5.0    0
4       1   (80,85]  3.5    0
5       2   (70,75]  4.5    1
6       3   (70,75]  4.5    0
7       3   (75,80]  1.0    0
8       4   (70,75]  4.0    0
9       4   (75,80]  2.0    1
10      5   (70,75]  3.0    0
11      5   (75,80]  5.0    0
12      5   (80,85]  5.0    0
13      6   (75,80]  3.1    0
14      6   (80,85]  2.0    1
15      7   (80,85]  3.0    1
16      8   (80,85]  3.1    0
\end{verbatim}
\normalsize
\end{column}
% 
\begin{column}{0.50\textwidth}
% Interpretation of variable names in {\tt lexis}:
\bi
\item[ ]
\item[ ] \verb|lex.id| = subject \\ $\quad$index 
in original \\ $\quad$data frame,
\item[ ] \verb|ageband| = ageband \\
            $\quad$ and  its limits, 
\item[ ] \verb|y_ik| = person-time slot 
        \\ $\quad$ 
       spent in ageband
\item[ ] \verb|d_ik| = indicator for \\
   $\quad$ event occurring \\
   $\quad$ in ageband.
\ei
\end{column}
\end{columns}
 
 \bigskip
Subject 1's follow-up time ($14.5\mbox{ y} = 1+ 5 + 5 + 3.5\mbox{ y}$)
is split into 4 agebands,
$\dots,$ subject 8 contributes only to 1 ageband.

\end{frame} 

%-------------------------------------

\begin{frame}[fragile]
 \frametitle{Tabulation of cases, rates etc. by ageband}

\ \\
Event indicators \& person-time slots
are summed over the rows of the split-expanded data
frame in categories of {\tt ageband}: 
\small
\begin{verbatim} 
>  D <- with(coh.A, tapply(d_ik, ageband, sum))
>  Y <- with(coh.A, tapply(y_ik, ageband, sum))
\end{verbatim}
\normalsize
Incidence rates ($I$), cumulative rates ($C$) and incidence proportions ($R$), 
the latter two by the end of each {\tt ageband}:
\small 
\begin{verbatim}
>  I <- 100*D/Y; C <- cumsum((D/Y)*5); R <- 1-exp(-C)
>  round(cbind(D,Y,I,C,R),3)[2:4, ]
        D    Y      I     C     R
(70,75] 1 21.0  4.762 0.238 0.212
(75,80] 1 16.1  6.211 0.549 0.422
(80,85] 2 16.6 12.048 1.151 0.684
\end{verbatim}
\normalsize

\end{frame} 

%-------------------------------------------

\begin{frame} \frametitle{Example: The Diet Study (see C\&H)}
\ \\
A cohort of 337 men in three occupational groups in England, \\ 
aged 30 to 67 y at entry, recruited in '50s and '60s, \\ 
followed-up until mid '70s for incidence of CHD events. 

\bigskip
Risk factors of interest, measured by dietary survey at entry.
\bes
\mbox{\tt energy} & = & \mbox{total energy intake (kcal/d)}, \\
\mbox{\tt energy.grp} & = & \mbox{{\tt energy} dichotomized:} \\ 
 & { } &  \mbox{ 1: ``$<=$2750 KCals'', 2: ``$>$2750 KCals''},\\ 
\mbox{\tt fat} & = & \mbox{fat intake (g/d)}, \\
\mbox{\tt fibre} & = & \mbox{dietary fibre intake (g/d),} \\
      {  } & & \mbox{\tt height, weight, bmi, }etc.
\ees

\end{frame} 

%----------------------------------------------------

\begin{frame} \frametitle{Important dates and outcome event}
\ \\
The data set {\tt diet} in {\tt Epi} contains three dates: 
\bes
\mbox{\tt dob} & = & \mbox{date of {\bf birth}}, \\
\mbox{\tt doe} & = & \mbox{date of {\bf entry} into follow-up}, \\
\mbox{\tt dox} & = & \mbox{date of {\bf exit}, end of follow-up}.
\ees
These are given in format {\tt yyyy-mm-dd} but implicitly stored as 
\textit{number of days since 1.1.1970}. 

In addition, the outcome event is represented by
\bes
\mbox{\tt chd} & = & \mbox{indicator for {\bf status} at exit:} \\ 
& { } & \quad \mbox{ 1 = CHD event occurred, 0 = censored} . 
\ees
\end{frame} 

%---------------------------------------------------

\begin{frame}[fragile]
\frametitle{Data \texttt{diet}: creating a \texttt{Lexis} object}
\ \\
First convert all dates into fractional calendar years
using function \texttt{cal.yr()} in \texttt{Epi} 
\small
\begin{verbatim}
diet <- transform(diet, doe = cal.yr(doe), 
     dox = cal.yr(dox), dob = cal.yr(dob) )
\end{verbatim}
\normalsize
Convert the data frame into a \texttt{Lexis} object.  
\small
\begin{verbatim}
> dietL <- Lexis( entry = list(age = doe-dob),
+                  exit = list(age = dox-dob),
+           exit.status = chd, data = diet )
\end{verbatim}
\normalsize
In the nexty step the \texttt{Lexis} object is splitted
according to 3 agebands (y): $30-<50$, $50-<60$, $60-<70$

\end{frame}


\begin{frame}[fragile]
\frametitle{Splitting the \texttt{Lexis} object into agebands}
\small
\begin{verbatim}
dietA <- splitLexis(dietL, br = c(30,50,60,70), 
                   time.scale = "age")
dietA$ageband <- timeBand(dietA, "age", "factor")
dietA$y_ik <- dietA$lex.dur ; dietA$d_ik <- dietA$lex.Xst
\end{verbatim}
\scriptsize
\begin{verbatim}
    id    dob    doe    dox    y chd   energy.grp ageband  age y_ik d_ik
1  102 1939.2 1976.0 1986.9 10.9   0 <=2750 KCals (30,50] 36.9 10.9    0
2   59 1912.5 1973.5 1982.5  9.0   0 <=2750 KCals (60,70] 61.0  9.0    0
3  126 1920.0 1970.2 1984.2 14.0   1 <=2750 KCals (50,60] 50.2  9.8    0
4  126 1920.0 1970.2 1984.2 14.0   1 <=2750 KCals (60,70] 60.0  4.2    1
5   16 1906.7 1969.4 1970.0  0.6   1 <=2750 KCals (60,70] 62.7  0.6    1
6  247 1918.5 1968.2 1979.5 11.3   1 <=2750 KCals (30,50] 49.7  0.3    0
7  247 1918.5 1968.2 1979.5 11.3   1 <=2750 KCals (50,60] 50.0 10.0    0
8  247 1918.5 1968.2 1979.5 11.3   1 <=2750 KCals (60,70] 60.0  1.0    1
\end{verbatim}
\normalsize
Properties of the original data frame and the expanded object:
\small
\begin{verbatim}
> str(diet) 
'data.frame':   337 obs. of  17 variables:
> str(dietA)
Classes Lexis and data.frame  729 obs. of 25 variables
\end{verbatim}

\normalsize
\end{frame}


%-------------------------------------------------

\begin{frame}[fragile]
 \frametitle{Relevelling of \texttt{energy.grp} and some tabulations}
\ \\
The \texttt{energy.grp} variable is relevelled such that
``high energy'' is taken as the reference or
``unexposed'' category and ``low energy'' as the ``exposed'' one.
 \small
 \begin{verbatim} 
dietA$eg2 <- Relevel( dietA$energy.grp, 
           ref = ">2750 KCals" )
\end{verbatim}
\normalsize
Tabulation of cases, person-years and rates: \small
\begin{verbatim}
tab.ae <- stat.table( list( ageband, eg2),
      list( D = sum(d_ik), Y = sum(y_ik),
            I = ratio(d_ik, y_ik, 1000) ),
       margin = T, data = dietA )
print(tab.ae, digits= c(sum=0, ratio=1))
\end{verbatim}
\normalsize
\end{frame} 


%------------------------------------------------

\begin{frame}
[fragile]
 \frametitle{Rates by ageband and energy intake}
 
\begin{columns}
\begin{column}[t]{0.50\textwidth}
{\scriptsize
\begin{verbatim}
  ------------------------------------ 
            -----------eg2----------- 
 ageband       >2750  <=2750   Total  
               KCals   KCals          
 ------------------------------------ 
 (-Inf,30]        NA      NA      NA  
  ...                                     
 (30,50]           4       2       6  
                 622     381    1003  
                 6.4     5.2     6.0                                      
 (50,60]           6      12      18  
                1128     979    2107  
                 5.3    12.3     8.5                                      
 (60,70]           8      14      22  
                 794     699    1493  
                10.1    20.0    14.7                                      
 (70,Inf]         NA      NA      NA
 ...                                      
 Total            18      28      46  
                2544    2059    4604  
                 7.1    13.6    10.0  
 ------------------------------------ 
\end{verbatim}
}
\end{column}
%\hfill
\begin{column}[t]{0.45\textwidth}

\ \\
\small
Crude rate ratio\\
\verb|> tab.ae[3, 6, 2] /| \\
\verb|+ tab.ae[ 3, 6, 1] | \\
\verb|[1] 1.921747| \\

\medskip
Rate ratios % for ``low'' vs. ``high`'' 
 by ageband: \\
\verb|> IRs <- tab.ae[3, 2:4, 2]/|\\ 
\verb|+        tab.ae[3, 2:4, 1]|\\         
\verb|> round(IRs,2)|\\
\verb|30-<50 50-<60 60-<70|\\ 
\verb|  0.82   2.30   1.99| 
\bi
\item Low intake risky?
\item No effect in young?
\ei
\normalsize
\end{column}
\end{columns}
\end{frame} 


%--------------------------------------------

\begin{frame} \frametitle{Poisson model on age and exposure}
\ \\
Let $D_{kj}, Y_{kj}$, and $I_{kj}$ be cases, p-years \& rate
in ageband $k$ \& exposure category $j$ 
(1=``unexposed'', 2=``exposed''). \\
Piecewise Exp-model in both exposure categories assumed:
$$ \lambda_{kj} = \mbox{ theoretical rate in cell }kj . $$
Theoretical { rate ratio} $\rho_k = \lambda_{k2}/\lambda_{k1}$, \\ comparing 
exposed {\it vs.} unexposed.
\bi
\item[(a)] What are the ``true'' values of $\rho_k$?
\item[(b)] Can we assume $\rho_k = \rho$, same rate ratio in all agebands?
\item[(c)] What is the value of the common rate ratio $\rho$?
\ei
\end{frame} 

%---------------------------------------------

\begin{frame} \frametitle{Poisson model (cont'd)}
\ \\
Assuming common rate ratio the true rates are modelled
$$ \log \lambda_{kj} = \alpha_k + \beta_j 
= \sum_{k=1}^K \alpha_k A_k + \sum_{j=1}^2 \beta_j X_j , $$
where $A_k$ and $X_j$ are indicator (1/0) variables for level $k$
of ageband and level $j$ of exposure.
In exponential form
$$ \lambda_{kj} = \exp( \alpha_k + \beta_j ) 
= e^{\alpha_k} e^{\beta_j} . $$
Set $\beta_1 = 0$ (``unexposed'' as reference)
 $\Rightarrow$ Interpretation:
\bes
\alpha_k & = & \log(\lambda_{k1})
 = \mbox{ log-rate of unexposed in ageband }k\\
\beta_2 & =  & \log(\lambda_{k2}/\lambda_{k1}) = \log(\rho)  
 = \mbox{log-common rate ratio}
\ees
\end{frame} 


%----------------------------------------------

\begin{comment}
\begin{frame} \frametitle{Generalized linear model on rates}
\ \\
Our log-linear model has already
\bi
\item {\bf linear predictor} $\ \ \eta_{kj}\
 = \ \sum_{k}\ \alpha_k A_k + \sum_{j}\ \beta_j X_j $,
\item {\bf link function} $\ \ \log(\lambda_{kj}) = \eta_{kj}$.
\ei 
To get a {\bf generalized linear model} (GLM) we still need
\bi
\item {\bf distribution} or {\bf family} for the outcome variable.
\ei
We can take either $D_{kj}$ or $I_{kj}$ as the outcome assuming
$$ D_{kj} \sim \mbox{Poisson}(\lambda_{kj} Y_{kj}), \quad \mbox{or} \quad
   I_{kj} \sim \frac{1}{Y_{kj}} \mbox{Poisson}(\lambda_{kj} Y_{kj}) $$ 
\end{frame} 

%---------------------------------------------------

\begin{frame} [fragile] \frametitle{Poisson regression on rates}
\ \\
This GLM is also called (log-linear) Poisson regression model.


With the {\it expected} no. of cases 
$\mu_{kj} = \lambda_{kj} Y_{kj} = E(D_{kj})$
the model can be expressed
$$ \log(\mu_{kj}) = \log(Y_{kj}) + \alpha_k + \beta_j $$
in which $\log(Y_{kj})$ is an {\it offset} term.
This must be specified when the
model is fitted for numbers of cases $D_{kj}$.

Alternatively, if rates $I_{kj}$ are taken as the outcome, 
the person-years $Y_{kj}$ must be declared as {\it weights}
\end{frame} 

\end{comment}

\begin{frame}[fragile]

\frametitle{Fitting the Poisson model}
\ \\
Use function {\tt glm()} on the expanded data frame:
\begin{verbatim}
> m.ea <- glm( d_ik/y_ik ~ ageband + eg2, 
+        fam = poisson, w = y_ik, data = dietA )
\end{verbatim}
\small
\begin{verbatim}
> round(ci.lin(m.ea, Exp=T)[ , -(3:4)], 4 )
                Estimate StdErr exp(Est.)   2.5%  97.5%
(Intercept)      -5.4033 0.4390    0.0045 0.0019 0.0106
ageband(50,60]    0.3027 0.4721    1.3535 0.5366 3.4145
ageband(60,70]    0.8456 0.4613    2.3294 0.9431 5.7535
eg2<=2750 KCals   0.6233 0.3027    1.8651 1.0306 3.3753
\end{verbatim}
\normalsize
The estimated rate ratio for ``low'' vs. ``high''
energy consumption, adjusted for age, is thus
1.87 [1.03 to 3.38], only slightly lower than the 
unadjusted one 1.92 [1.06 to 3.47].

\end{frame}

\begin{frame}
\frametitle{Concluding remarks}
 
\bi
\item Modelling could continue from this to include other confounders,
  continuous covariates, interactions, {\it etc.} 
 \medskip
\item
Agebands may well be much narrower than in our example. With infinitely
narrow bands Poisson regression equals the famous Cox model.
\medskip
\item
Splitting by many time scales ({\it e.g.} age, calendar time, time since first exposure, {\it etc.}) simultaneously and the corresponding data frame expansion is straightforward using these tools. More about this 
in the next lecture. 
\ei

\end{frame}
\end{document}
\begin{comment}
%-------------------------------------------

\begin{frame}[fragile]
 \frametitle{Steps before fitting Poisson regression}
\ \\
A grouped data frame is formed from tables (p. 32); \\
{\tt ageband} and {\tt energy.g2} are already defined as {factors}:\\
{\tt > py <- as.data.frame(as.table(pyrs)) \\
> ca <- as.data.frame(as.table(cases)) \\
> names(py)[3] <- "Y"   \# name of the 3th column changed\\
> names(ca)[3] <- "D"   \# to describe the true content\\
> diet.grp <- merge(ca,py) \# merging the two files into one\\
> diet.grp{\$}I <- 1000*diet.grp{\$}D/diet.grp{\$}Y ; diet.grp }\\
\scriptsize
\begin{verbatim}
  ageband     energy.g2  D         Y         I
1 [30,50]   <2750kcal/d  2  380.9528  5.249995
2 (50,60]   <2750kcal/d 12  979.3374 12.253182
3 (60,70]   <2750kcal/d 14  699.1403 20.024593
4 [30,50]  >=2750kcal/d  4  622.3819  6.426922
5 (50,60]  >=2750kcal/d  6 1127.6961  5.320582
6 (60,70]  >=2750kcal/d  8  794.1602 10.073535
\end{verbatim}
\normalsize

\end{frame} 
%-----------------------------------

\begin{frame}[fragile] 
\frametitle{Fitting Poisson model (cont'd)}
\ \\
{\tt > options(show.signif.stars=FALSE) \# no stars needed!}\\
Ready to fit! The intercept is removed from model formula.\\
$\Rightarrow$ {\it treatment contrast} without intercept for {\tt a.band}

{\tt > fit1 <- glm(D $\sim$ - 1 + ageband + energy.g2,\\
+ \ \ \ family=poisson(link=log),offset=log(Y),data=diet.grp)\\
> summary(fit1)   \# some parts of output skipped }

\scriptsize
\begin{verbatim}
Coefficients:
                       Estimate Std. Error z value Pr(>|z|)
ageband[30,50]          -4.7800     0.4314  -11.08   <2e-16
ageband(50,60]          -4.4773     0.2625  -17.06   <2e-16
ageband(60,70]          -3.9345     0.2420  -16.26   <2e-16
energy.g2 >=2750kcal/d  -0.6233     0.3026   -2.06   0.0394
\end{verbatim}
\normalsize
\end{frame} 
 
%------------------------------------

\begin{frame} \frametitle{Results on ratio scale}
\ \\
User-supplied function {\tt expcoef.wald}\\
$\mbox{  - }$ extracts estimated coefficients
\& their covariance matrix,\\ 
$\mbox{  - }$
 computes estimates \& 95\% CIs on the ratio scale.\\
These confidence limits are based on {\it Wald statistic}.

{\tt > expcoef.wald <- function(mfit)\{RateRatio<- exp(coef(mfit)) \\
> SE.coef <- sqrt(diag(vcov(mfit))) \# std. errors \\
> CI95.lo = RateRatio*exp(-1.96*SE.coef)\\
> CI95.up = RateRatio*exp(+1.96*SE.coef)\\
> round(cbind(RateRatio, CI95.lo, CI95.up),4)\} }\\
\end{frame} 

%---------------

\begin{frame}[fragile]
 \frametitle{Results on ratio scale (cont'd)}
\ \\
More accurate limits are computed, based on {\it profile likelihood}, 
by function {\tt confint} found in {\tt library(MASS)}.\\
{\tt > expcoef.prof <- function(mfit) \{ \\
+  round(exp(cbind(RateRatio=coef(mfit),confint(mfit))),4)\} }\\
Confidence intervals by both methods appear similar: \\
{\tt > cbind(expcoef.wald(fit1), expcoef.prof(fit1))}\\
{\scriptsize
\begin{verbatim}
                       RateRatio CI95.lo CI95.up RateRatio  2.5 % 97.5 %
ageband[30,50]            0.0084  0.0036  0.0196    0.0084 0.0032 0.0179
ageband(50,60]            0.0114  0.0068  0.0190    0.0114 0.0065 0.0184
ageband(60,70]            0.0196  0.0122  0.0314    0.0196 0.0118 0.0305
energy.g2 >=2750kcal/d    0.5362  0.2963  0.9702    0.5362 0.2912 0.9622
\end{verbatim}
}
 Some evidence for energy intake level being
inversely associated with CHD risk. -- Causal interpretation?
\end{frame} 

%------------------------------
\begin{frame} [fragile]
\frametitle{Goodness of fit}
\ \\
If the model is correct and the data are not sparse\\
$\bullet$ {\it deviance residuals}
 should behave almost as N(0,1)-variates,\\
$\bullet$ {\it residual deviance} Dev as a Chi-square variate
with\\
$\bullet$ {\it residual degrees of freedom} \\
$\ \ \ \ \mbox{     df } = \mbox{ no. of observations $-$ no.
of parameters}$.\\
Hence we can expect E(Dev) = df. Here the fit seems OK:

\scriptsize 
\begin{verbatim} 
Deviance Residuals: 
       1         2         3         4         5         6  
-0.72024   0.25780   0.08817   0.67228  -0.33971  -0.11418  
    Null deviance: 8702.4122  on 6  degrees of freedom
Residual deviance:    1.1734  on 2  degrees of freedom     AIC: 31.474

> round(fit1$fitted.values,2) # fitted numbers of cases in cells
    1     2     3     4     5     6 
 3.20 11.13 13.67  2.80  6.87  8.33 
\end{verbatim}
\normalsize
\end{frame} 

%----------------------------------------

\begin{frame} \frametitle{Alternative modes of fitting}
\ \\
Fitting of the above model can also be based on
\bi
\item individual time-slot records,
\item rates instead of cases as the outcome, or/and
\item parametrization with the intercept.
\ei
All these alternatives are actually more general, and
are applied on the expanded data frame: \\
{\tt > fit2 <- glm(chd/y.ik $\sim$ ageband + energy.g2, \\
+ family=poisson(link=log), weights=y.ik)} 

     
\end{frame} 

%-------------------------

\begin{frame}[fragile] \frametitle{Linear predictor with intercept}
\ \\
This model formula is equivalent to
$\lambda _{kj} = \alpha + \nu_k + \beta_j. $ \\
Putting both $\nu _1 = 0$ and $\beta _1=0$ (default in R) implies
\bes
 \alpha & = & \mbox{intercept }= \log(\lambda _{11})
   = \alpha_1 =\mbox{ log-rate in the {\it corner cell}}\\
 \nu _k & = & \log(\lambda _{k1}/\lambda _{11} )
 = \alpha_k - \alpha_1 \\ 
  & = & \mbox{{\it treatment contrast} of
 ageband $k$ to ageband 1.}
\ees

{\tt > summary(fit2) \# two coef's already familiar!}
\scriptsize
\begin{verbatim}
Coefficients:
                        Estimate Std. Error z value Pr(>|z|)    
(Intercept)              -4.7800     0.4312 -11.085   <2e-16 
ageband(50,60]            0.3027     0.4713   0.642   0.5207    
ageband(60,70]            0.8456     0.4606   1.836   0.0664   
energy.g2  >=2750kcal/d  -0.6233     0.3024  -2.061   0.0393   
\end{verbatim}
\normalsize
\end{frame} 

%----------------------

\begin{frame}[fragile] \frametitle{Goodness of fit}
\bi
\item[ ] These data are {\it sparse}: no. of failures is 1 in 46 records and
0 in 683. Hence, the asymptotic properties of the deviance
residuals and the residual deviance do not hold.
\item[$\Rightarrow$] Parts of output from {\tt summary(fit2)} poorly interpretable:
\\
{\scriptsize
\begin{verbatim}
Deviance Residuals: 
    Min       1Q   Median       3Q      Max  
-0.6254  -0.4312  -0.3247  -0.2050   2.8937  
    Null deviance: 321.09  on 728  degrees of freedom
Residual deviance: 311.44  on 725  degrees of freedom
AIC: Inf
\end{verbatim}
}
\item[ ]
However, fitted no's of cases $\widehat\mu_{kj}$,
fitted rates $\widehat\lambda _{kj}$ \& dev. residuals
in {\tt ageband$\times$energy.g2} cells can be computed:\\

\end{frame} 

\end{document}
%------------------------------------------

\begin{frame}[fragile]
 \frametitle{Comparing observed and fitted values}

{\tt > muh <- as.vector(tapply(fit2{\$}fitted.values*y.ik,\\
+ \ \ \ list(ageband, energy.g2),sum)) \# fitted no. of cases\\
> lamh <- 1000*muh/as.vector(pyrs)    \# fitted rates\\
> attach(diet.grp)   \#  make use of a ready dataframe\\
> r.dev <- sign(D-muh)* \# deviance residuals\\      
+ \ \ \  sqrt( 2* ( ifelse(D > 0, D*log(D/muh), 0) - (D-muh)))\\
> cbind(diet.grp[,1:2],round(cbind(D,muh,I,lamh,r.dev),2))}\\
{\scriptsize
\begin{verbatim}
  ageband     energy.g2  D   muh     I  lamh r.dev
1 [30,50]   <2750kcal/d  2  3.20  5.25  8.40 -0.72
2 (50,60]   <2750kcal/d 12 11.13 12.25 11.36  0.26
3 (60,70]   <2750kcal/d 14 13.67 20.02 19.56  0.09
4 [30,50]  >=2750kcal/d  4  2.80  6.43  4.50  0.67
5 (50,60]  >=2750kcal/d  6  6.87  5.32  6.09 -0.34
6 (60,70]  >=2750kcal/d  8  8.33 10.07 10.49 -0.11
\end{verbatim}
}
Compare 3 first {\tt lamh} values with the ratio estimates (p. 40)
\end{frame} 

%-----------------------------------------

\begin{frame} \frametitle{Interaction -- modification of rate ratio}
\ \\
Consider a model with ageband $\times$ exposure interaction, represented
by {\it interaction parameters} $\delta_{kj}$
$$ \lambda_{kj} = \alpha + \nu_k + \beta _j + \delta _{kj}. $$
Set $\delta_{k1} = \delta_{1j} = 0$ for all $k,j$. Interpretation:
\bes
\alpha & = & \log(\lambda _{11}) = \mbox{ log-rate in the corner cell}\\
\beta_2 & = & \log(\lambda_{12}/\lambda_{11}) = \log(\rho_1) = \mbox{ log-rate-ratio in ageband 1}\\
\nu_k & = & \log( \lambda_{k1}/\lambda_{11}) 
  = \mbox{ log-ratio: ageband $k$ vs. ageband 1}\\
\delta_{k2} & = & \log( \lambda_{k2}/\lambda_{k1}) - \log(\lambda_{12}/\lambda_{11}) 
   = \log(\rho_k/\rho_1) \\
 & = & \mbox{log-ratio of exposure-rate-ratios btw agebands $k$ \& 1}
\ees
\end{frame} 

%--------------------------------------------

\begin{frame}
[fragile] \frametitle{Fitting a model with interaction}
\ \\
The main-effect model (with intercept added) updated\\
{\tt > fit3 <- update(fit1, $\sim$ . + 1 + ageband:energy.g2)}\\
{\scriptsize
\begin{verbatim}
> cbind(Estimate=summary(fit3)$coef[,1],expcoef.wald(fit3))
                                        Estimate RateRatio CI95.lo CI95.up
ageband[30,50]                        -5.2495275    0.0052  0.0013  0.0210
ageband(50,60]                        -4.4019696    0.0123  0.0070  0.0216
ageband(60,70]                        -3.9107941    0.0200  0.0119  0.0338
energy.g2 >=2750kcal/d                 0.2022680    1.2242  0.2244  6.6787
ageband(50,60]:energy.g2 >=2750kcal/d -1.0364709    0.3547  0.0500  2.5165
ageband(60,70]:energy.g2 >=2750kcal/d -0.8893175    0.4109  0.0611  2.7643
\end{verbatim} 
}

This model is {\it saturated}: $\widehat\lambda_{kj} = \widehat I_{kj}$.  Interpretation:
\bes 
\widehat\alpha & = & -5.2495 = \log(\widehat\lambda_{11}) = \log(5.2 \mbox{ per 1000 y}) = \log(I_{11})\\
\widehat\beta_2 & = & 0.2023 = \log(1.22) = \log(\widehat\rho_1)
 = \log( \widehat\lambda_{12}/\widehat\lambda_{11})\\ 
\widehat\delta_{32} & = & -0.889 = \log(0.41) = \log( \widehat\rho_3/\widehat\rho_1 )
 = \log(0.50/1.22)   
\ees
\end{frame} 

%-------------------------------------------
\begin{frame}[fragile]
 \frametitle{Adding another, continuous covariate}
\ \\
With time-slot expanded data possible to include continous
covariates \& their ind'l values without need to categorize. \\
{\it E.g.} body mass index BMI = weight (kg) / height$^2$ (m$^2$), \\
linear (centred and scaled) and quadratic terms: \\
{\tt > bmi1 <- (weight/(height/100){\^{}}2-25)/5; bmi2<-bmi1{\^{}}2 \\ 
> fit4 <- update(fit2, $\sim$ . + bmi1) \\
> cbind(Estimate=summary(fit4){\$}coef[,1],expcoef.prof(fit4))} 

{\scriptsize
\begin{verbatim}

                         Estimate RateRatio  2.5 % 97.5 %
(Intercept)            -4.8933591    0.0075 0.0026 0.0170
ageband(50,60]          0.5027221    1.6532 0.6591 5.0128
ageband(60,70]          1.0608805    2.8889 1.1829 8.6386
energy.g2 >=2750kcal/d -0.6881887    0.5025 0.2686 0.9193
bmi1                    0.2856270    1.3306 0.8293 2.1196
\end{verbatim}
}

\end{frame} 


%----------------------------------------

\begin{frame} \frametitle{Continuous covariates (cont'd)}
\ \\
The model fitted above is specified for each subject $i$:
$$ \lambda_{ik} = \alpha + \sum_{k=1}^3 \nu_k A_{ik} 
+  \sum_{j=1}^2 \beta_j X_{ij} + \gamma Z_{i1} , $$
where $A_{ik}$ is indicator for {\tt ageband} $k$, 
$X_{ij}$ that for {\tt energy.g2} level $j$, and
$Z_{i1}$ is the centred and scaled BMI value. 
With $\nu_1 = \beta_1 = 0$,
the interpretation of selected estimates is:
\bes
  e^{\widehat\alpha} & = & 7.5 \mbox{ per 1000 y } 
  = \mbox{ fitted rate among} \\ 
 & { } & \mbox{unexposed in ageband 1 with BMI 25 kg/m$^2$} \\
 \widehat\gamma & = & 0.2856 = \log(1.33)  \\
 & = & \mbox{log-rate-ratio for a change of 5 kg/m$^2$ in BMI}
\ees

\end{frame} 

%-----------------------------------------

\begin{frame} \frametitle{Analysis of deviance}
\ \\
Evaluation of "significance" of a new term $V_p$ (factor or variable)
added to model formula containing terms \\ $V_1, \dots, V_{p-1}$ can be
based on $\Delta$Dev = Dev$_0$ - Dev$_1$, i.e. 
 {\it difference in (residual) deviance} between these 
two {\it nested models} $M_0$ and $M_1$.

If the smaller model $M_0$ is "true", i.e. the
regression coefficients for term $V_p$ are 0, $\Delta$Dev 
is approximately Chi-square-distributed with d.f. being
the difference in the residual d.f. between the models.
\end{frame} 

%------------------------------------

\begin{frame}[fragile] \frametitle{Analysis of deviance table by R}
\ \\
The terms in {\tt fit4} are fitted in the order:
 {\tt ageband, energy.g2, bmi1}. Function {\tt anova}
obeys this order when reporting $\Delta$Dev values ({\tt = Deviance}
column)
between successive models.

{\tt > anova(fit4)}
{\scriptsize
\begin{verbatim}
            Df Deviance Resid. Df Resid. Dev
NULL                          718    313.275
ageband      2    6.543       716    306.732
energy.g2    1    4.069       715    302.664
bmi1         1    1.413       714    301.251
\end{verbatim} 
}

Hence, {\tt energy.g2} is "significant" upon {\tt ageband} but {\tt bmi1}
is not "significant" upon {\tt ageband} and {\tt energy.g2}.  \\
{\bf Warning!} {\it Do not apply this mechanically, and} \\ {\it not at all 
when evaluating confounding!}

\end{frame} 

\end{document}

%--------------------------------------