\setbeamertemplate{footline}[frame number]
\let\olditem\item
\renewcommand{\item}{%
\olditem\vspace{\fill}}   
% additions 
% This is a file of useful extra commands snatched from
% Michael Hills, David Clayton, Bendix Carstensen & Esa Laara.
%

% Commands to draw observation lines on follow-up diagrams
%
% Horizontal lines
%

% exit time with failure, bullet
\newcommand{\hfail}[1]{\begin{picture}(250,5)
       \put(0,0){\line(0,1){2.5}}
      \put(0,0){\line(0,-1){2.5}}
      \put(0,0){\line(1,0){#1}}
      \put(#1,0){\circle*{5}}
   \end{picture}}

% exit time with censoring, open circle
\newcommand{\hcens}[1]{\begin{picture}(250,5)
         \put(0,0){\line(0,1){2.5}}
      \put(0,0){\line(0,-1){2.5}}
      \put(0,0){\line(1,0){#1}}
%      \put(#1,0){\line(0,1){2.5}}
%      \put(#1,0){\line(0,-1){2.5}}
% BxC Changed this to an open circle instead of a line
      \put(#1,0){\circle{5}}
   \end{picture}}

%
% Diagonals for Lexis diagrams
%
\newcommand{\dfail}[1]{\begin{picture}(250,250)
      \put(0,0){\line(1,1){#1}}
      \put(#1,#1){\circle*{5}}
   \end{picture}}

\newcommand{\dcens}[1]{\begin{picture}(250,250)
      \put(0,0){\line(1,1){#1}}
%      \put(#1,#1){\line(0,1){2.5}}
%      \put(#1,#1){\line(0,-1){2.5}}
% BxC Changed this to an open circle instead of a line
      \put(#1,#1){\circle{5}}
   \end{picture}}

%
% Horizontal range diagrams
%
\newcommand{\hrange}[1]{\begin{picture}(200,5)
     \put(0,0){\circle*{5}}
     \put(0,0){\line(1,0){#1}}
     \put(0,0){\line(-1,0){#1}}
   \end{picture}}

%
% Tree drawing
%
\newcommand{\tree}[3]{\setlength{\unitlength}{#1}\begin{picture}(0,0)
   \put(0,0){\line(3,2){1}}
   \put(0,0){\line(3,-2){1}}
   \put(0.81,0.54){\makebox(0,0)[br]{\footnotesize #2\ }}
   \put(0.81,-0.54){\makebox(0,0)[tr]{\footnotesize #3\ }}
\end{picture}}

\newcommand{\wtree}[3]{\setlength{\unitlength}{#1}\begin{picture}(0,0)
   \put(0,0){\line(1,1){1}}
   \put(0,0){\line(1,-1){1}}
   \put(0.8,0.8){\makebox(0,0)[br]{\footnotesize #2\ }}
   \put(0.8,-0.8){\makebox(0,0)[tr]{\footnotesize #3\ }}
\end{picture}}

\newcommand{\ntree}[3]{\setlength{\unitlength}{#1}\begin{picture}(0,0)
   \put(0,0){\line(2,1){1}}
   \put(0,0){\line(2,-1){1}}
   \put(0.8,0.4){\makebox(0,0)[br]{\footnotesize #2\ }}
   \put(0.8,-0.4){\makebox(0,0)[tr]{\footnotesize #3\ }}
\end{picture}}

%
% Other commands
%
\newcommand{\prob}[0]{\text{\rm Pr}}
\newcommand{\nhy}[0]{_{\oslash}}
\newcommand{\true}[0]{_{\text{\rm \tiny T}}}
\newcommand{\hyp}[0]{_{\text{\rm \tiny H}}}
% \newcommand{\mpydiv}[0]{\stackrel{\textstyle \times}{\div}}
% Changed to slightly smaller symbols
\newcommand{\mpydiv}[0]{\stackrel{\times}{\scriptstyle \div}}
\newcommand{\mie}[1]{{\it #1}}
\newcommand{\mycircle}[0]{\circle*{5}}
\newcommand{\smcircle}[0]{\circle*{1}}
\newcommand{\corner}[0]{_{\text{\rm \tiny C}}}
\newcommand{\ind}[0]{\hspace{10pt}}
\newcommand{\gap}[0]{\\[5pt]}
\renewcommand{\S}[0]{section~}
\newcommand{\blank}[0]{$\;\,$}
\newcommand{\vone}{\vspace{1cm}}
\newcommand{\ljust}[1]{\multicolumn{1}{l}{#1}}
\newcommand{\cjust}[1]{\multicolumn{1}{c}{#1}}
\newcommand{\mean}{\text{\rm Mean}}
\newcommand{\transpose}{^{\mbox{\tiny T}}}
\newcommand{\histog}[5]{\rule{1mm}{#1mm}\,\rule{1mm}{#2mm}\,\rule{1mm}{#3mm}\,\rule{1mm}{#4mm}\,\rule{1mm}{#5mm}}
\newcommand{\pmiss}{P_{\mbox{\tiny miss}}}
%
% Below is BxCs commands inserted
%
\newcommand{\bc}{\begin{center}}
\newcommand{\ec}{\end{center}}

\newcommand{\bd}{\setlength{\parskip}{1ex} \begin{description}}
\newcommand{\ed}{\end{description} \setlength{\parskip}{2ex}}
\newcommand{\bdx}{\begin{description}} % Bendix' description macros
\newcommand{\edx}{\end{description}}

\newcommand{\bix}{\begin{itemize}}  % these are Bendix' itemizing macros
\newcommand{\eix}{\end{itemize}}
\newcommand{\bi}{\setlength{\parskip}{1ex} \begin{itemize}} % Esa's item macros 
\newcommand{\ei}{\end{itemize} \setlength{\parskip}{2ex}} 

\newcommand{\bn}{\begin{equation}}
\newcommand{\en}{\end{equation}}
\newcommand{\be}{\begin{enumerate}}
\newcommand{\ee}{\end{enumerate}}
\newcommand{\bes}{\begin{eqnarray*}}
\newcommand{\ees}{\end{eqnarray*}}
\newcommand{\p}{\text{\rm P}}
\newcommand{\pmat}[1]{\text{\rm P}\left\{#1\right\}}
\newcommand{\ptxt}[1]{\text{\rm P}\left\{\text{\rm #1}\right\}}
\newcommand{\E}{\text{\rm E}}
\newcommand{\V}{\text{\rm V}}
\newcommand{\BLUP}{\text{\rm BLUP}}
% \newcommand{\var}{\mbox{Var}} changed by Esa to
\newcommand{\var}{\mbox{var}}
% \newcommand{\cov}{\mbox{Cov}} changed by Esa to
\newcommand{\cov}{\mbox{cov}}
% \newcommand{\corr}{\mbox{Corr}} changed by Esa to
\newcommand{\corr}{\mbox{corr}} 
%\newcommand{\var}{\text{\rm var}}
%\newcommand{\cov}{\text{\rm cov}}
%\newcommand{\corr}{\text{\rm corr}}
\newcommand{\se}{\text{\rm s.e.}}
\newcommand{\sd}{\text{\rm std}}
\newcommand{\erf}{\text{\rm erf}}
\newcommand{\odds}{\text{\rm odds}}
\newcommand{\bin}{\text{\rm binom}}
\newcommand{\half}[1]{\frac{1}{#1}}
% \newcommand{\td}[0]{\stackrel{\textstyle \times}{\div}}
% Changed to slightly smaller symbols
\newcommand{\td}[0]{\stackrel{\scriptstyle \times}{\scriptstyle \div}}
\newcommand{\logit}{\text{\rm logit}}
\newcommand{\link}{\text{\rm link}}
\newcommand{\spn}{\text{\rm span}}
\newcommand{\OR}{\text{\rm OR}}
\newcommand{\CI}{\text{\rm CI}}
\newcommand{\RR}{\text{\rm RR}}
\newcommand{\QR}{\text{\rm QR}}
\newcommand{\QD}{\text{\rm QD}}
\newcommand{\ER}{\text{\rm ER}}
\newcommand{\EM}{\text{\rm EM}}
\newcommand{\EF}{\text{\rm EF}}
\newcommand{\RD}{\text{\rm RD}}
\newcommand{\AC}{\text{\rm AC}}
\newcommand{\AF}{\text{\rm AF}}
\newcommand{\PAF}{\text{\rm PAF}}
\newcommand{\SR}{\text{\rm SR}}
\newcommand{\SMR}{\text{\rm SMR}}
\newcommand{\SEL}{\text{\rm SEL}}
\newcommand{\CMF}{\text{\rm CMF}}
\newcommand{\pvp}{\text{\rm PV}$+$}
\newcommand{\pvn}{\text{\rm PV}$-$}
\newcommand{\R}{\textsf{R}}
%\newcommand{\gap}[0]{\\[5pt]} 
%\newcommand{\blank}[0]{$\;\,$}
% Conditional independence sign from Philip Dawid
\newcommand{\cip}{\mbox{$\perp\!\!\!\perp$}}
%%% Commands to comment out parts of the text
\newcommand{\GLEM}[1]{}
\newcommand{\FORGETIT}[1]{}
\newcommand{\OMIT}[1]{}
%%% Insert output from program in small text 
%%% (requires package verbatim)
\newcommand{\insout}[1]{
\scriptsize
\renewcommand{\baselinestretch}{0.8}
\verbatiminput{#1}
\renewcommand{\baselinestretch}{1.0}
\normalsize
}
% From Esa:        
\newcommand{\T}{\text{\rm \small{T}}}
\newcommand{\id}{\text{\rm id}}
\newcommand{\Dev}{\text{\rm Dev}}
\newcommand{\Bin}{\text{\rm Bin}}
\newcommand{\probit}{\text{\rm probit}}
\newcommand{\cloglog}{\text{\rm cloglog}}
%\newcommand{\EF}{\text{\rm EF}}
\newcommand{\SE}{\text{\rm SE}}
\newcommand{\IP}{\text{\rm IP}}
\newcommand{\+}{\tiny +}