% typeset using LaTeX2e
\documentclass[handout,12pt]{beamer}
\mode<presentation>
{
 \usetheme{Singapore}
 \setbeamercovered{transparent}
}

% making handouts
%\usepackage[overlay]{textpos}
%\usepackage{pgfpages}
%\pgfpagesuselayout{4 on 1}[letterpaper, border shrink = %5mm, landscape]
%\mode<presentation>{
%  \setbeamercovered{invisible}
%}
% end of making handouts
%\usepackage{rotating}
\usepackage{graphicx}
%\usepackage{mslapa}
\usepackage{amsmath}
%\usepackage[all]{xy}
\usepackage{amssymb,graphicx}
%\input psfig.sty
\usepackage{booktabs}
\usepackage{graphpap}
\usepackage{verbatim}
%\usepackage{amsbsy}
%\graphicspath{E:/figures/}
\graphicspath{{C:/Users/janne.pitkaniemi/figures/}}

\setbeamertemplate{footline}[page number]
\DeclareGraphicsExtensions{.pdf, .jpg, .png,.jpeg}
% \DeclareGraphicsExtensions{.eps}

\newcommand{\Rlogo}[1]{\includegraphics[#1]{Rlogo}}

\input{usefulesa}
\usepackage[labelformat=empty]{caption}
%\captionsetup[figure]{labelformat=empty}

% \newcommand{\R}{\bf \textsf{R}}
\parskip\medskipamount

\title{Epidemiologic Data Analysis using R \\
Part 6: Analysis of case-control studies}  % : \\ Analysis of Follow-up Studies}
% \Rlogo{height=1em} }

\author{Janne Pitk\"aniemi \\ (Esa L{\"a}{\"a}r{\"a})}

\institute{Finnish Cancer Registry, Finland,   
 \texttt{<janne.pitkaniemi@cancer.fi>} \\
 (University of Oulu, Finland,   
 \texttt{<esa.laara@oulu.fi>}) }


\date{University of Tampere \\Faculty of Social Sciences \\ % University of Tampere/Postgraduate training,
Feb 26- Apr 9  2018}

%\begin{center} \Rlogo{height=2em} \end{center}

\AtBeginSubsection[]
{
  \begin{frame}<beamer>
    \frametitle{Outline}
    \tableofcontents[currentsection,currentsubsection]
  \end{frame}
}

%\beamerdefaultoverlayspecification{<+->}

\begin{document}
\definecolor{darkgreen}{rgb}{0,.5,0}

\begin{frame}
    \titlepage
\end{frame}

%-------------------------------------------

\begin{frame}
\frametitle{Contents}
 
\bi
\item[1.] Case-control designs 
\medskip
\item[2.] Exposure odds ratio (EOR)
  and its interpretation
\medskip
\item[3.] Estimation of EOR by logistic regression
\medskip
\item[4.] Matched case-control studies
\ei

Main R functions to be covered
\bi
\item {\tt glm()} 
\ei
\end{frame}

%----------------------------------------------------------------------
\begin{frame} \frametitle{Case-control design}

% {\bf\large Simplified description}
\bi
\item
From given study  population (base pop'n) are selected 
all or a random sample of
\bi
{\normalsize
\item $D$ {\bf cases}, or individuals with the \\
  disease being diagnosed during certain period
\medskip
\item $C$ {\bf controls}, or ``healthy'' individuals at risk.
}
\ei
\item
Exposure to risk factor $X$ and other covariates \\
  assessed in cases and chosen controls.
 \medskip
 \item
 To increase efficiency and remove confounding,
the sampling of controls is often {\it stratified} or individually {\it matched} for age, gender,
place of residence, {\it etc}.
 \ei
\end{frame} 
%------------------------------------
\begin{frame} 
\frametitle{Exposure odds ratio (EOR)}
\ \\
With binary risk factor $X$ the results are summarized:

\begin{center}
\begin{tabular}{lccc}
\toprule
Exposure         &        Cases        &    Controls & Total\\
\midrule
yes $(X=1)$        &        $D_{1}$ &        $C_{1}$  & $T_1$  \\
no $(X=0)$         &        $D_{0}$  &        $C_{0}$  & $T_0$  \\
\midrule  
Total        &                $D $  &  $C$  & $T $ \\
\bottomrule
\end{tabular}
\end{center}

\ \\
Common effect measure:
\bi
\item {\bf exposure odds ratio}
      $$  \text{EOR}  
          =  \frac{ D_{1} / D_{0} } { C_{1} / C_{0} }  
          = \frac{ D_{1} C_{0} } { D_{0} C_{1} }        
      $$
\ei
\end{frame} 


%-----------------------------
 

\begin{frame}
 \frametitle{Precision in EOR}
\ \\
Standard error of log(EOR),  95\% {error factor} (EF) \& \\
 95\% confidence interval (CI) for 
the associated parameter:
\bes
 \SEL & = & \sqrt{ \frac{1}{D_1} + \frac{1}{D_0} 
    + \frac{1}{C_1} + \frac{1}{C_0} } \\ & & \\
 \EF & = & \exp \{ 1.96 \times \SEL \} \\ & & \\
 \CI & = & [ \text{EOR} / \EF ,\ \text{EOR} \times \EF ] . 
\ees
NB. Random error depends inversely on numbers of cases and controls.
\end{frame} 

%--------------------

\begin{frame} \frametitle{What parameter is estimated by EOR?}
\ \\
The answer depends on 
\bi
\item type of base population, from which cases emerge
\bi
{\normalsize
\item[--] closed population or {\bf cohort}, or 
\item[--] open or {\bf dynamic population},
}
\ei
\item time dimensionality  \\
-- longitudinal or cross-sectional
\medskip
\item sampling principle of controls:
% Various sampling strategies (see e.g. Silva 1999, ch 9):
\bi
{\normalsize
\item[(A)] case-noncase sampling
 (epidemic ca-co study)
\medskip 
\item[(B)] case-cohort sampling
\medskip
\item[(C)] density sampling (incl. nested case-control study)
}
\ei
\ei
\end{frame} 


\begin{frame}
\frametitle{Sampling controls from a longitudinal base}

\pause
Simplified ideal situation:\\
 Complete follow-up 
of a cohort of initially healthy subjects with 
no losses during a fixed risk period.

\pause
\begin{center}
{\normalsize
\setlength{\unitlength}{0.08cm}
\begin{picture}(135,50)
\thicklines
\linethickness{0.3mm}

%mittlere Box

\put(40,45){\line(6,-1){55}}
\put(40,5){\framebox(55,40){}}

%Achsenbeschriftung
\put(45,-1.5){\makebox(25,5){\small Time ($t$)}}
\put(25,-1){\makebox(25,5){\small Start}}
\put(85,-1){\makebox(25,5){\small End}}
\put(70,2.5){\vector(1,0){20}}

\pause
%linker Text
\put(0,5){\makebox(40,40){\parbox{7cm}{\begin{center}
						   {\small{\bf B}:  Initially at risk}\\
						%	(disease-free)\\ (C)\\
							{\small $(N)$}
							\end{center} }}}
\pause
\put(40,15){\makebox(55,10){\parbox{11cm}{ \begin{center}
						{\small {\bf C}: Currently at risk $(N_t)$}
							\end{center} }}}
\put(67.5,25){\vector(0,1){15}}
\put(67.5,15){\vector(0,-1){9}}							
%rechter Text
\pause
\put(100,30){\makebox(40,15){\parbox{7cm}{\begin{center}
					{\small		New cases\\of disease\\ $(D)$}
							\end{center} }}}
\pause							
\put(100,5){\makebox(40,20){\shortstack{\strut\small
       {\bf A}: Still at risk \\ \strut
						%	(disease-free)\\ \strut 	(A)\\
						 \strut $(N - D)$ }}}


\end{picture}
}
\end{center}

Possible sampling frames: {\bf A, B} and {\bf C}
\end{frame} 

%---------------------------------

\begin{frame}
\frametitle{Sampling schemes or designs for controls}

\medskip
\pause
{\bf A:  Case-noncase sampling}\\
\pause
           \bi
           \item
           Controls chosen from those $N-D$ 
           subjects % cohort members
                            still at risk (healthy) 
                            % , or free from the outcome 
                    \underline{\it at the end}
				    of the follow-up. 
	     \ei \pause
\pause   	     
{\bf B:  Case-cohort sampling}: % ({\it juhtkohortuuring})  
\pause\medskip
	     \bi
\item The control group or {\bf subcohort}   
           is a random sample of the whole cohort $(N)$  
				    \underline{\it at the start} of the follow-up.
           \ei
\pause           	     
{\bf B:  Density sampling}: \\ % (\textit{tihedusvalik}):  \\
         \pause\medskip  \bi 
         %  \item[--] Also called % {\it density} sampling, 
           % {\it risk-set} s., 
           % {\it time-matched} s., or
         %  {\it concurrent} sampling.
			  %        or {\it dynamic} s.
           \item  
           Controls drawn 
            \underline{\it during the follow-up}
              from those \\ currently at risk. 
           \medskip
           \item {\bf Nested case-control design (NCC)}
            % (\textit{pesastatud \dots}):
             \\
         A set of controls is sampled from
         the {\it risk set} 
          %  (alive \& yet uncensored)
	%			  for the outcome 
			  \underline{\it at each time $t$ of diagnosis}
                           of a new case,
	%			     i.e. have not yet contracted the disease by then.
	 %\medskip
   %       \item[$\bullet$]  applicable both in cohorts 
   %        and in dynamic populations.
   %        \pause\medskip
   %       \item[$\bullet$] in dynamic populations, 
   %       time-matching not always exact.
	     \ei
\end{frame}



\begin{frame}
\frametitle{EOR in case-noncase sampling design}

\bi
\item
In the traditional or epidemic case-control study
% or case-noncase design, 
the controls are selected from
those still healthy at the end of the
risk period, during which cases are collected.
\medskip
\pause
\item
In this design EOR estimates the {\bf risk odds ratio} 
$$ \psi = \frac{ \text{odds of dis. in the exp'd} }
                     {   \text{odds of dis. in the unexp'd} } 
     = \frac{ \pi_1/(1-\pi_1) }{ \pi_0/(1-\pi_0) }, $$
  where $\pi_1$ and $\pi_0$ are the {risks} of disease in the exposed and unexposed groups, estimable from a corresponding cohort study by
  incidence proportions $R_1$ and $R_0$. 
\pause
\medskip 
\item 
{\bf NB.}  $\psi \approx \pi_1 / \pi_0 = \phi$, {\it i.e.} close to {\bf risk ratio}, when risks $\pi_1$ and $\pi_0$ are low % ($< 0.1$) 
\pause
= the ``rare disease assumption''.              
\ei

% \pause
% \medskip
% \item Cross-sectional: EOR estimates \textbf{prevalence odds ratio}.
\end{frame} 

%--------------------

\begin{frame} \frametitle{Density sampling}

\bi
\item New incident cases occurring during given study period 
      are identified from the base population.
\medskip      
\item controls are randomly chosen from the population  
      at risk at various times in the period 
% $t_i$ when a case is diagnosed
      (sometimes only once).
\medskip      
 \item For chronic disease studies this design is the most popular,
\item Logically the only possibility in open populations,
\ei

{\bf Nested case-control study}: {\it time-matched} selection:
\bi
\item one or more (rarely over $5$)
 controls chosen from the \\ population at risk 
 at each time $t_d$ when a new case is diagnosed. 
 \ei
\end{frame} 


%--------------------

\begin{frame} \frametitle{EOR in density sampling}

% With this design EOR {\it does not} estimate
% the risk odds ratio $\psi$ (like in case-noncase sampling) but 
\bi
\item
 In a full cohort
study the true hazard ratio
$\rho = \lambda_1 / \lambda_0 $ 
is estimated by the incidence rate ratio 
$$  \text{IR} = \frac{I_1} {I_0} = \frac{D_1 / D_0 }{ Y_1 / Y_0} . $$
\item
In a case-control study with density sampling 
the {\bf exposure odds} among controls
$C_1/C_0$ estimates the {exposure odds} $Y_1/Y_0$, i.e.
the distribution of  person-years % into exposure groups
 in the base population.
\medskip
\item
Thus, the exposure odds ratio EOR
$$ \text{EOR} = \frac{D_1 / D_0 }{ C_1 / C_0} \approx 
  \frac{D_1 / D_0 }{ Y_1 / Y_0} = \text{IR} $$
is a consistent estimator of the true hazard rate ratio
  $\rho$ without any rare-disease assumption.
\ei  
\end{frame} 


%----------------------------------
\begin{frame}
 \frametitle{Example. Alcohol use and oesophageal cancer}
 (Tuyns et al 1977, see {\bf B\&D} 1980).

\bi
\item
205 new cases of cancer identified in a French province 
during two years, 
and 770 randomly sampled population
controls $\Rightarrow$ Density sampling
\item
{\bf NB.} No stratification or matching for age in design \\ 
$\Rightarrow$
Too many young controls in relation to few cases \\
$\Rightarrow$  inefficient! 
\medskip\item
Exposure of interest: Daily consumption of alcohol.
\medskip\item
In the following table the data are summarized by dichotomized 
exposure and stratified by age group.
\medskip\item
In R the data are found: {\tt data(esoph)}
\ei 
\end{frame} 

%----------------------------------
\begin{frame} \frametitle{Example: Results stratified by age} 
{\scriptsize
\begin{center}
\begin{tabular}{lcccc}
\toprule
Age & Exposure $\geq 80$ g/d & Cases & Ctrls & EOR \\
\midrule
25-34 & yes         &  1 & 9 & $\infty$  \\
 { } & no                & 0 & 106 & { } \\
\cmidrule{2-5} 
35-44 & yes         &  4 & 26 &  5.05 \\
 { } & no                & 5 & 164 & { } \\
\cmidrule{2-5} 
45-54 & yes         &  25 & 29 & 5.67 \\
 { } & no                & 21 & 138 & { } \\
\cmidrule{2-5} 
55-64 & yes         &  42 & 27 & 6.36 \\
 { } & no                & 34 & 139 & { } \\
\cmidrule{2-5} 
65-74 & yes         &  19 & 18 & 2.58 \\
 { } & no                &  36 & 88 & { } \\
\cmidrule{2-5}
75-84 & yes         &  5 &  0 & $\infty$ \\
 { } & no                & 8 & 31 & { } \\
\bottomrule 
Total & yes   & 96 & 109 & 5.64 \\
{ }  & no    & 104 & 666 & {(crude)} \\
\bottomrule
\end{tabular}
\end{center}
}
\end{frame} 

\begin{frame}\frametitle{Example: (cont'd)} 
\ \\

Modification?
\bi
\item Stratum-specific EOR$_k$s somewhat variable.
\medskip
\item Random error in some of them apparently great (especially in the youngest and 
	the oldest age groups).
\ei

Confounding?
\bi
\item Is exposure associated with age in the study population?
\medskip 
\item Look at variation in the age-specific prevalences of exposure among controls.
\medskip
\item Adjustment for age is generally reasonable.
\ei
\end{frame} 

%-------------------------------


%----------------------------------------------------------------------
\begin{frame} \frametitle{Model for stratified data}
\ \\
{\it Random part}: 

Conditional on total number of subjects 
$$T_{jk} = D_{jk} + C_{jk}$$ 
in each level $j$ $(j=1, 2)$ of
exposure variable $X$ 
and level $k$ $(k=1,\dots K)$ of
covariate $Z$ we assume 
$$ D_{jk} \sim  \mbox{Binomial}(T_{jk};\ p_{jk}) , $$
where $p_{jk}$ is the ``probability of being a case''
in a group of cases \& controls defined by $X$ and $Z$.

\end{frame} 
%----------------------------------

\begin{frame} \frametitle{Model for stratified data (cont'd)}

\medskip
{\it Systematic part \& logit link}:
\begin{center} 
$ \text{logit}( p_{jk} )
 = \log \left( \frac{\displaystyle p_{jk}}
  { \displaystyle 1 - p_{jk} } 
   \right ) = \alpha + \beta X + \gamma_k , $
\end{center}   
\bi
\item[ ]$X$ = exposure, 1: 'exposed'; 0: 'unexposed',
\item[ ]$\alpha = \text{logit}(p_{11}) = $  
                  log of ``pseudo baseline odds'',
\item[ ]$\beta$ = logarithm of exposure odds ratio,  
\item[ ]$\quad  =    \log(\rho)$, 
     logarithm of true rate ratio $\rho$
\item[ ] \quad\quad with density sampling,
\item[ ]$\gamma_k$ = logarithm of 
   rate ratio btw levels $k$ and  1  of  $Z$.
\ei
Hence $e^{\beta} = \rho$ is the common rate ratio 
for the exposure effect assumed constant over the levels of $Z$. 
\end{frame} 
%-------------------------------

\begin{frame}[fragile] 
\frametitle{Example. Estimation by {\tt glm()}}

Input of data
{\small
\begin{verbatim}
D <- c(0,1, 5,4, 21,25, 34,42, 36,19, 8,5) # no. of cases
C <- c(106,9, 164,26, 138,29, 139,27, 88,18, 31,0) # controls
T <- D + C         # cell totals
\end{verbatim}
}
Generation and naming of the levels for factors describing 
age group and alcohol exposure
{\small
\begin{verbatim}
agr <- gl(6,2,12)  # 6 levels for age factor
levels(agr) <- c("25-34", "35-44", "45-54", 
                "55-64", "65-74", "75-84")
alc <- gl(2,1,12)  # 2 levels for alcohol factor
levels(alc) <- c("0-79g/d", "80+g/d")
\end{verbatim}
}
\end{frame}
% The estimated rate ratio $e^{1.67} = 5.31$ [95\% 3.7 to 7.7]
%    Null deviance: 211.608  on 11  degrees of freedom
%Residual deviance:  11.041  on  5  degrees of freedom
% AIC: 65.369
% \end{verbatim}

\begin{frame}[fragile]
\frametitle{Example. Estimation by {\tt glm()}}

{\small
\begin{verbatim}
> data.frame( agrn = as.numeric(agr), agr, 
              alcn = as.numeric(alc), alc, D, C, T)
   agrn   agr alcn     alc  D   C   T
1     1 25-34    1 0-79g/d  0 106 106
2     1 25-34    2  80+g/d  1   9  10
3     2 35-44    1 0-79g/d  5 164 169
4     2 35-44    2  80+g/d  4  26  30
5     3 45-54    1 0-79g/d 21 138 159
6     3 45-54    2  80+g/d 25  29  54
7     4 55-64    1 0-79g/d 34 139 173
8     4 55-64    2  80+g/d 42  27  69
9     5 65-74    1 0-79g/d 36  88 124
10    5 65-74    2  80+g/d 19  18  37
11    6 75-84    1 0-79g/d  8  31  39
12    6 75-84    2  80+g/d  5   0   5
\end{verbatim}
}

\end{frame}

\begin{frame}[fragile]
\frametitle{Example. Estimation by {\tt glm()} cont'd)}

Crude estimation
{\small
\begin{verbatim}
> mod1 <- glm( D/T ~ alc, fam = binomial, w = T)
> round(ci.lin(mod1, Exp=T)[ , -(3:4)], 4)

            Estimate StdErr exp(Est.)   2.5%  97.5%
(Intercept)  -1.8569 0.1054    0.1562 0.1270 0.1920
alc80+g/d     1.7299 0.1752    5.6401 4.0006 7.9515
\end{verbatim}
}

Estimation adjusted for age \\
{\small
\verb|> mod2 <- update(mod1, . ~ . + agr)|
} 

Goodness-of-fit evaluation \\
{\small
\verb|> c( mod2$deviance, mod2$df.res )|\\
\verb|[1] 11.04118    5|
}

\medskip

\end{frame}


\begin{frame}[fragile]
\frametitle{Example. Estimation by {\tt glm()} (cont'd)}

Estimation results after adjusting for age
{\small
\begin{verbatim}
> round(ci.lin(mod2, Exp=T)[ , -(3:4)], 4)

            Estimate StdErr exp(Est.)   2.5%    97.5%
(Intercept)  -5.0543 1.0094    0.0064 0.0009   0.0461
alc80+g/d     1.6699 0.1896    5.3116 3.6630   7.7022
agr35-44      1.5423 1.0659    4.6753 0.5788  37.7668
agr45-54      3.1988 1.0232   24.5022 3.2984 182.0171
agr55-64      3.7135 1.0185   40.9966 5.5688 301.8094
agr65-74      3.9669 1.0231   52.8196 7.1112 392.3239
agr75-84      3.9622 1.0650   52.5723 6.5193 423.9520

\end{verbatim}
}
\end{frame}

%-------------------------------------------------

\begin{frame}
\frametitle{Matched case-control study}
\ \\
Matching
\bi
\item For each case choose 
    1 or more (rarely over $4$) controls with
      same age ({\it eg.} within 1 year, 
      or in the same 5-year ageband), 
      sex, place of living, {\it etc.} 
      \medskip
\item Implies stratification in design: 
each matched case-control set forms one stratum.
    \medskip
\item Improves efficiency of the study \& estimation of 
effect parameters, if matching factors are 
   strong determinants of outcome.
\ei
\end{frame} 

%--------------------------------

\begin{frame}
\frametitle{Matched case-control study (cont'd)}
\ \\
Some principles
\bi
\item Impractical to match on many other covariates than those mentioned,
\medskip
\item Matching on a correlate $Z$ of risk factor $X$ of interest, which is not 
	causal determinant of outcome \\
	 $\Rightarrow$ overmatching, loss of efficiency.
	\medskip
\item {\it Counter-matching}: Choose controls which are different from
	case w.r.t. $Z$, close correlate of $X$ \\
	 $\Rightarrow$ increases efficiency.
\ei
\end{frame} 


%--------------------------------

\begin{frame}
\frametitle{Matched case-control study (cont'd)}
\ \\
\bi
\item
{\it Matched design $\Rightarrow$ matched analysis!}
\medskip
\item
Ignoring matching in analysis my lead to biased results.
\medskip
\item
Matching factors must always be accounted 
for in estimating the rate ratios of interest. 
\medskip
\item
With very close matching (based {\tt e.g.} on sibship, neighbourhood)
 use {\it  conditional logistic regression} modelling \\
 -- function {\tt clogit()} in package {\tt survival}
\ei

\end{frame} 

\begin{frame}
\frametitle{Concluding remarks}
\bi

\item
Analysis using {\tt glm()} on individual data records from 
an unmatched study proceeds similarly as for grouped data.
\medskip
\item
Matched design $\to$ matched analysis by \texttt{clogit()}.
\medskip
\item
More complicated designs, like counter-matched and two-phase studies,
require specialized methods and progamming.
\medskip
\item
Case-cohort design: Use function {\tt coxph()} in package {\tt survival}
but adjust standard errors {\it etc.} appropriately.  
\ei
\end{frame}

\end{document}
%-------------------------------------


\end{document}








\begin{frame}\frametitle{Example.}  
\ \\
Use of oestrogens and endometrial cancer 
(Mack et al. NEJM 1976; 294: 1262-67, see also Breslow \& Day 1980).

63 women with endometrial cancer diagnosed in 1971-75, aged 57 to 83 y, 
living in retirement community near Los Angeles were identified as case group.  
\end{frame} 

%-------------------------------------

\begin{frame}\frametitle{Example (cont'd).}  
\ \\
Each case woman was matched to 4 control women who 
\bi
\item were alive and living in same retirement community at the time case was 
	diagnosed,
\item were born within 1 year of the case's time of birth
\item had same marital status,
\item had entered the community at about same time,
\item had not have hysterectomy prior to date of diagnosis of the case.
\ei
\end{frame} 
%----------------------------------------------

\begin{frame}\frametitle{Example (cont'd)} 
Dataset {\tt bdendo.dta} includes following variables
\bi \item[]
\bi \item[]
\bi
\item[{\tt set} =] number of case-control set (stratum),
\item[{\tt d} =] case indicator: 1= case, 0 = control.
\ei
\ei
\ei
Exposure variables of interest
\bi \item[]
\bi \item[]
\bi
\item[{\tt est} =] use of any estrogen: 1=''yes'', 0=''no'',
\item[{\tt cest} =] conjugate estrogen dose: \\
  		1 = none, 2= 0.1-0.299,  3= 0.3-0.625, 4=0.625+
\item[{\tt dur} =] duration (months) of conjugate estrogen therapy:\\
  		0 = none, 1= 1-11, 2=12-47, 3= 48-95, 4=96+ 
\ei
\ei
\ei
\end{frame} 

%--------------------------------

\begin{frame}\frametitle{Example (cont'd)}
\ \\[-30pt]
Some covariates
\bi \item[]
\bi \item[]
\bi
\item[{\tt age} =] age in years,
\item[{\tt agegrp} =] age group: 
            1 = 55-59, $\dots$ , 6=80-84 years,
\item[{\tt gall} =] history of gallbladder disease:  
          2=''yes'', 1=''no'',
\item[{\tt hyp} =] history of hypertension: 2=''yes'', 1=''no'',
\item[{\tt ob} =] history of obesity: 2=''yes'', 1=''no'',
\item[{\tt non} =] history of use of non-estrogen drugs: \\
   2=''yes'', 1=''no''.
\ei
\ei
\ei
\end{frame} 
%----------------------------------

\begin{frame}\frametitle{Stratified analysis}
\ \\
Mantel-Haenszel summary EOR$^{\MH}$ for risk factor(s) $X$ of interest 
similarly calculated as before, stratification being based
on individual case-control sets.

Main effects of the matching variables cannot be
estimated.

{\it Example.} Effect of {\tt est} adjusted for matching is estimated by
\bi
\item[.] mhodds d est, by(set)
\ei
\end{frame} 
%----------------------------------

\begin{frame}\frametitle{Stratified analysis}
\ \\
Result:
$\Rightarrow$ EOR$^{\MH} = 8.46 > 7.87$ = crude EOR
\end{frame} 
%----------------------------

\begin{frame}\frametitle{Modification in stratified analysis}
\ \\
Evaluation of effect modification 
for covariate $Z$ included in matching criteria:
\bi
\item[$\Rightarrow$] Estimate $\EOR^{\MH}$ separately for
	levels of $Z$ and inspect heterogeneity
	of these informally or by statistical test.
\ei
\end{frame} 
%----------------------------

\begin{frame}\frametitle{Modification in stratified analysis (cont'd)}
\ \\
{\it Example.} Modification by age: Calculate $\EOR^{\MH}$ for {\tt est}
separately for {\tt age} $< 70$ years and {\tt age} $\geq 70$.
\bi
\item[.] {\tt mhodds d est if agegrp $<$ 4, by(set)}
\ei

\bi
\item[.] {\tt mhodds d est if agegrp $>$ 3, by(set)}
\ei

\end{frame} 
%------------------------------------

\begin{frame}\frametitle{Modification (cont'd)}
\ \\
Modification due to covariates not matched for:
very inefficient to evaluate.

{\it Example.} Modification by {\tt gall}:
\bi
\item[.] {\tt mhodds d est if gall==1, by(set)}
\ei
\end{frame} 
%------------------------------------

\begin{frame}\frametitle{Modification (cont'd)}
\ \\
\bi
\item[.] {\tt mhodds d est if gall==0, by(set)}
\ei
Severe loss of information.

Modelling is called for!
\end{frame} 

%------------------------------------------------

\begin{frame}\frametitle{Conditional logistic regression}
\ \\[-20pt]
To take matching into account,
common {\it unconditional} logistic model
$$ \logit( \pi_{jk} ) =
     \alpha + \beta x + \gamma_k  $$
would necessarily include separate parameter $\gamma_k$
for all $-$ one case-control sets $k$.
\bi
\item[$\Rightarrow$] no. of parameters $\rightarrow$ too large,
\item[$\Rightarrow$] bias in estimating exposure effect $\beta$.
\ei
Remedy: eliminate ''nuisance parameters'' $\alpha_k$ by {\it conditional logistic 
regression}.
\end{frame} 

%------------------------------------------------

\begin{frame}\frametitle{Conditional logistic regression (cont'd)}
\ \\
Contributions $L_k$ from each case-control set $k$ 
to the {\it conditional likelihood} used in ML estimation from each case-control 
set are of the form
\bi
\item[ ]
$L_k$ = ''{\it probability that the case subject is a case given that one set 
member is a case}''.
\ei
These conditional probabilities have complicated expressions. 
They depend on risk factor and covariates and parameter(s) of 
interest $\beta$, etc. but no more on $\gamma_k$s.

\end{frame} 
%---------------------------------

\begin{frame}\frametitle{Fitting in Stata}
\ \\[-20pt]
First adjusting for matching factors only
\bi
\item[.] {\tt  clogit d est, strata(set) or}
\ei
Model-adjusted EOR somewhat smaller than EOR$^{\MH}$

NB! Ignore again the {\tt Std.Err.} column
\end{frame} 
%---------------------------------

\begin{frame}\frametitle{Fitting (cont'd)}
\ \\[-20pt]
On logit scale:
\bi
\item[.] {\tt  clogit d est, strata(set) }
\ei

Here  {\tt Std.Err.} meaningful.
\end{frame} 

%-------------------------------

\begin{frame}\frametitle{Fitting (cont'd)}
\ \\[-20pt]
Further adjustment for {\tt gall} and evaluation of
modification by {\tt gall}
\bi
\item[.] {\tt xi: clogit d i.est i.gall, strata(set) or}
\ei

\end{frame} 

%-------------------------------

\begin{frame}\frametitle{Fitting (cont'd)}
\ \\[-20pt]
\bi
\item[.] {\tt xi: clogit d i.est i.gall, strata(set) or}
\ei
Strong negative modification! 

Interpretation?
\end{frame} 

%---------------------------------------
\end{document}